%%=============================================================================
%% Voorbeeldapplicatie
%%=============================================================================

\chapter{\IfLanguageName{dutch}{Voorbeeldapplicatie}{Example of an application}}
\label{ch:voorbeeldapplicatie}
In dit hoofdstuk wordt er stap voor stap beschreven hoe de opstelling is opgebouwd om de vergelijking te HANA tussen de verschillende build schedulers.
%TODO: figuren tussen haakjes zijn niet zeker of ze wel in het document moeten komen. Dit is duidelijk uitgelegd in onderstaande tekst.

    \section{Build scheduler server}
    \label{sec:build-scheduler-server}
    Amista heeft een Ubuntu server ter beschikking gesteld dat wordt gehost op Digital Ocean. De server wordt gebruikt om de build schedulers op te draaien en zo te vergelijken. Eerst moeten er enkele belangrijke zaken ingesteld worden alvorens aan de slag te gaan, zoals security en dergelijke.
    
        \paragraph{Digital Ocean}
        %TODO: uitleggen wat Digital Ocean is en doet.
        
        \paragraph{Ubuntu server}
        %TODO: Uitleggen welke kenmerken de Ubuntu server heeft.
        
        \paragraph{Installatie Ubuntu server}
        Eerst HANA we de Ubuntu server klaar voor gebruik om zo de nodige zaken te installeren.
        %TODO: In figuur EersteKeerInloggenOpUbuntuServer in de bijlagen wordt er getoond wat er moet gebeuren als er voor de eerste keer ingelogd wordt op de server.
        Via de root account wordt er via SSH ingelogd op de server. 
        
        SSH staat voor secure shell en is een software protocol dat voor een veilige verbinding (tunnel) zorgt tussen de client en de server. Het wordt gebruikt voor het configureren van een server, het beheren van netwerken en operating systems. Alle gegevens dat tussen beiden worden uitgevoerd zijn geëncrypteerd waardoor het moeilijker wordt voor hackers om de data te bemachtigen.
        
        Een server die gebruik maakt van SSH wordt ook wel een sshd server genoemd. Eens ingelogd op de sshd server moeten er enkele zaken aangepast worden aan de ssh configuratie in de file /etc/ssh/sshd\_config. Omdat we hier via de root gebruiker werken moet de property PermitRootLogin op yes staan. Dit zorgt ervoor dat de root gebruiker kan inloggen.
        StrictMode moet ook op yes staan, zo kan er niemand inloggen als de authenticatie documenten leesbaar zijn voor iedereen. Dit voor het beveiligen van configuratie documenten. %TODO: Deze configuratie kan u zien in figuur Changing-sshd_config in de bijlagen.
        
        De default manier om in te loggen via ssh is via een account en een paswoord, maar het is ook mogelijk om het account en het paswoord te vervangen door een private en een public key. Dit principe noemt key-based authentication en wordt vooral tijdens development en in scripts gebruikt of voor single sign-on. SSH genereert een private en een public key op de client wanneer deze stap wordt geconfigureerd. De private key moet veilig bewaard worden op de client computer. De public key moet doorgegeven worden aan de remote server. Wanneer de client wil inloggen op de server voert hij een request uit. De server maakt via zijn public key een bericht en stuurt dit als response door naar de client. De client leest het bericht aan de hand van zijn private key en stuurt dan een aangepaste response terug naar de remote server. De server valideert deze response. Bij een geldige private key zal er een goede response verstuurd worden, bij een ongeldige private key een foute response.
        In deze thesis gaat men ervan uit dat de client computer een ssh key heeft die gebruikt kan worden. %TODO: Zoals u in figuur SSHPrivateKeyOnClient kan zien heeft de client die gebruikt werd tijdens het schrijven van deze thesis enkel lees- en schrijfrechten voor de file id\_rsa, dit om het als secret te bewaren.
        De id\_rsa.pub is de publieke key van de client die op de server moet komen om zo de ssh validatie te voorzien, dit wordt ook wel een ssh session genoemd. Eens de ssh session geconfigureerd is zal het niet nodig zijn om via een paswoord in te loggen op de remote server via deze client.
        %TODO: In figuur CopyPublicKeyToServer in de bijlagen is te zien hoe de ssh session wordt opgezet tussen de client en de remote server voor de root user.
        Voor deze thesis en om veiligheidsredenen is het beter om enkel via key-based authenticatie in te loggen en het paswoord uit te sluiten.
        %TODO: In figuur sshd_configNoPassword is te zien welke aanpassingen in de sshd\_config file van de sshd server moeten gebeuren om het niet meer mogelijk te HANA om in te loggen via een paswoord. PasswordAuthentication en ChallengeResponseAuthentication moeten naar no verandert worden. PubKeyAuthentication moet naar yes verandert worden.
        Nu moet de sshd\_config file opgeslagen worden (\^x + y + enter) en de ssh daemon herstart worden door het commando sudo systemctl restart ssh in te geven.
        
        Om de remote server nog meer te beschermen tegen cyber aanvallen is het nodig om een firewall op te zetten. In deze voorbeeldapplicatie HANA we gebruik van de UFW Firewall. Dis staat voor Uncomplicated Firewall en is een gebruiksvriendelijke tool dat helpt om de iptables onder controle te houden om zo te zorgen dat bepaalde services toegelaten worden tot onze server.
        In Linux HANA ze gebruik van het protocol SSH via de service OpenSSH, deze heeft ook een profiel bij UFW.
        %TODO: In Figuur UFWFirewallSetup in de bijlagen is te zien hoe de firewall de SSH service toelaat. Het is enkel mogelijk om de server te bereiken via deze service. Later worden er uiteraard meerdere services toegelaten.
        
        Nu alle stappen voor de configuratie van de server gedaan zijn is het zeer makkelijk om in te loggen op de server.
        Het is hetzelfde als de eerste keer, maar nu vraagt de server niet meer naar een paswoord, maar gebruikt hij de ssh-key. Het is voldoende om 
        ssh root@188.166.61.128 te typen om in te loggen.
        Als je wil uitloggen is het nodig om in de command line van de server exit te typen.
    
    \section{Database}
    \label{sec:database}
    Binnen SAP wordt een HANA database aangeraden om te gebruiken. Momenteel is versie 2.0 van SAP HANA op de markt en deze versie biedt tal van extra mogelijkheden ten opzichte van de vorige versie. SAP HANA wordt zeer goed ondersteund door de andere programma's binnen SAP en wordt daarom ook wel veel gebruikt.
    Voor de database HANA we gebruik van een Multi-Target Application Project, dit is een template die SAP ons geeft en is een goede uitvalsbasis om te gebruiken in de voorbeeldapplicatie.
    Zoals eerder al aangegeven is een Source Code Repository van groot belang voor een CI/CD pipeline en development in het algemeen.
    %TODO: In figuur CreatingRepositoryForDatabaseOnBitbucket kan u waarnemen hoe een repository gemaakt wordt in Bitbucket.
    Eens de repository aangemaakt is heb je het webadres nodig om de clone te HANA op je lokale machine.
    
    De volgende stap is het project aanmaken. In de Web IDE voor HANA development is het belangrijk om eerst enkele instellingen aan te passen.
    %TODO: Alle nodige features die nodig zijn kan u terug vinden in figuur ChangeFeaturesForHANA. Nu is het ook belangrijk om je Git account te connecteren met de Web IDE door het Git email adres en Git username in te voeren in de Git Settings.
    Zoals eerder vermeld HANA we het project aan de hand van het Multi-Target Application Project template. na de creatie van het project is het nodig om een build uit te voeren. (figuur InitialBuildHANAProject)
    Om het project aan het account op Bitbucket te linken klikt u op het project met de rechter muisknop, gaat u naar Git en dan Initialize Local Repository. (figuur InitializeLocalRepositoryHANA)
    Nu linken we de gemaakte repository aan het project door via rechter muisklik op het project op Git en dan Set Remote te klikken. Hier moet je de gekopieerde bitbucket link plakken in het veld voor URL. (figuur ConfigureGitRepositoryHANA) Na het ingeven van het juiste wachtwoord, is het nodig om op OK te klikken wanneer het Changes Fetched window opent.
    Na deze stap is het project gelinkt met de repository op Bitbucket. Het maken van het project heeft voor changes gezorgd in de repository, deze moeten eerst gecommit en gepusht worden. %TODO: In figuur InitialCommitHANA is te zien wat er allemaal moet gebeuren om een commit en push te doen naar de repository.
    
    Een realistische opstelling van een CI/CD voorbeeldapplicatie start bijna nooit vanaf nul, het bouwt meestal voort op bestaande, geschreven software.
    Daarom zal er in deze voorbeeldapplicatie manueel een basis gelegd worden, zo kan er aan de hand van een build scheduler voortgebouwd worden op deze geschreven software.
    De bescheiden database waar we naartoe willen gaan bestaat uit twee entiteiten: een Artiest en een Album. Een artiest kan meerdere albums hebben, maar een album kan maar tot één artiest behoren. De entiteit Artiest heeft volgende properties: ID, Naam, JaarVanOorsprong en Stad. Album heeft ID, Naam, Beschrijving, Genre, Jaar en Studio als properties. Als basis maken we de entiteit Artiest enkel met ID, Naam en JaarVanOorsprong. De rest zal later toegevoegd worden aan de hand van de pipeline.
    %TODO: Eerst maken we een HANA Database Module zoals in figuren MakenHANADatabaseModule-1, MakenHANADatabaseModule-2 en MakenHANADatabaseModule-3 te zien zijn. Daarna wordt een HDB CDS Artifact gemaakt. Dit is een Core Data Services document dat defenities bevat om objecten te creëeren in de database. Hoe de Artiest gedefinieerd wordt kan u zien in figuren MakenCDS-1, MakenCDS-2, MakenCDS-3 en MakenCDS-4.
    Eens deze gegevens ingevoerd zijn moeten we alle veranderingen toevoegen aan de commit om dan te pushen naar de repository.
    
    Om de models te gebruiken dat gecreëerd zijn, moeten we een tweede module toevoegen aan de HANA database: een Node.js module om de data als OData service te kunnen gebruiken.
    Deze module implementeert XSJS en XSODATA die op hun beurt zorgen voor de transformatie van het data model en de bereikbaarheid naar de buitenwereld. het is nodig om op de data CRUD (Create, Read, Update en Delete) operaties uit te voeren en als OData naar buiten gaat, dit is de taak van XSODATA. XSJS zorgt dan weer voor de integratie met SAPUI5. Het laat toe dat SAPUI5 applicaties de data kan lezen en kan bewerken.
    Zoals u in %TODO: figuur HANANodeJsFeature ziet moeten de instellingen aangepast worden. De feature Tools For Node.js Development moet beschikbaar zijn om te gebruiken in het project. Dit moet gesaved worden en het project moet opnieuw geladen worden. De volgende stap is de module maken door op het project met de rechtermuisknop te klikken, new en dan Node.js Module kiezen. Geef het een goede naam (js is good practice) en zorg ervoor dat Enable XSJX support aangevinkt is alvorens op Finish te klikken, zoals te zien is in %TODO: figuur MakenHANANodeJsModule-1, MakenHANANodeJsModule-2 en MakenHANANodeJsModule-3.
    %TODO: Databeveiliging is zeer belangrijk. Binnen HANA wordt er gebruik gemaakt van UAA (User Account en Authentication) om de Node.js module te beveiligen. Deze service moet samen met de database module en de HDI container, achter de db module, toegevoegd worden als resource zoals te zien valt in figuur MakenHANANodeJsModule-4.
    Om een OData service te maken moeten volgende stappen gebeuren: in de js-module in de lib folder, moet een xsodata-file gemaakt worden in een nieuwe folder, xsodata genaamd. De link met de data moet in deze nieuwe file gemaakt worden. Later zal er ook de associatie moeten gemaakt worden tussen Artiest en Album. Deze stappen zijn te zien in %TODO: figuren MakenHANANodeJsModule-5 en MakenHANANodeJsModule-6.
    De volgende stap is de xsjs service maken door in de lib folder een nieuwe folder, xsjs genaamd, te maken met de nieuwe hdb.xsjs file. Zie %TODO: figuur MakenHANANodeJsModule-7. De inhoud van deze nieuwe file is te zien in %TODO: figuur MakenHANANodeJsModule-8.
    De nieuwe Node.js module moet gebuild worden door op de module met de rechtermuisklik te klikken, Run, Run As en dan Node.js Application te kiezen.
    
    
     
    
    
    
    
    
    
    
    