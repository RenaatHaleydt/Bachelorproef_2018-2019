%%=============================================================================
%% Methodologie
%%=============================================================================

\chapter{\IfLanguageName{dutch}{Methodologie}{Methodology}}
\label{ch:methodologie}

Deze thesis gaat aan de hand van een vergelijkende studie op zoek naar de beste build scheduler voor de specifieke set-up die ze bij Amista hanteren. De omgeving waar de build-scheduler overweg mee moet kunnen ziet er als volgt uit: een SAPUI5 webapplicatie met een SAP HANA database draaiend via Node.js en alles gehost op SAP Cloud Platform. Er worden aan de hand van criteria enkele build-schedulers gekozen die worden opgezet in bovenstaande omgeving, de stappen die hierbij komen kijken worden zorgvuldig uitgeschreven en in een handleiding gegoten. Zo is deze proof-of-concept reproduceerbaar en heeft Amista een mooi voorbeeld om een CI/CD pipeline op te zetten.

    \section{Vergelijkende studie van de build schedulers}
    We beginnen met een klein stukje theorie over de build scheduler, om nadien tot de vergelijking over te gaan. Hier worden ook de verschillende criteria die Amista aangaf als belangrijk en minder belangrijk opgedeeld in twee delen: de functionele en de niet-functionele requirements. Binnen beide categorieën wordt er nog een opsplitsing gemaakt tussen: must-haves, should-haves en nice-to-haves.
    Met deze informatie kunnen we al een eerste vergelijking maken tussen de build-schedulers die vandaag op de markt te vinden zijn. Als resultaat krijgen we een long list waar de tools naast elkaar worden gezet en vergeleken kunnen worden aan de hand van criteria. 
    Uit deze eerste vergelijking nemen we de beste tools eruit om te testen in een realistische omgeving. Dit zal in hoofdstuk \ref{ch:proof-of-concept} gebeuren.

        \subsection{Build Scheduler}
        In Hoofdstuk~\ref{ch:ci-cd-cd} wordt er kort even aangehaald wat een build scheduler is en doet. Om hier toch een beter beeld van te krijgen, kaarten we nog een stukje theorie aan, om nadien aan de technische vergelijking te beginnen.
        De belangrijkste taak van een build scheduler is het uitvoeren van de builds. Met de principes van Continuous Integration in het achterhoofd weten we dat deze build uitgevoerd wordt na elke commit in de version control repository. Maar in praktijk zijn er twee verschillende categorieën: polling builds en scheduler builds.\\
        Bij polling builds wordt er, meestal op een zeer korte tijd zoals elke minuut bijvoorbeeld, gekeken of er veranderingen zijn aangebracht in de repository. Wanneer de build scheduler een verandering opmerkt zal hij automatisch de build uitvoeren.\\
        Bij scheduler builds wordt er, meestal op een dagelijkse basis, naar de veranderingen in de repository gekeken. Bij veranderingen worden de builds automatisch uitgevoerd. Deze manier is niet helemaal in regel met de principes van CI/CD omdat er op een dagelijkse basis feedback wordt voorzien in plaats van onmiddellijk na de commit, zoals bij polling builds.
        
        Iedere developer weet dat testen een cruciale rol hebben binnen software. Het automatiseren van de tests gebeurd door een testing tool, %TODO: zoals Karma vaak gebruik wordt binnen SAPUI5 development. Deze tool zorgt ervoor dat de QUnit (Unit tests binnen SAPUI5) en OPA tests (Integration tests binnen SAPUI5) geautomatiseerd worden. Continuous Integration en Continuous Delivery draait allemaal rond het uitvoeren van de testen en het feedback geven over de resultaten ervan. Het is dan ook de taak van de build scheduler om voor een uitstekende samenwerking te zorgen met de gebruikte testing tool.
        
        Het is ook de taak van de build scheduler om naadloos samen te werken met de gebruikte version control tool. Hij moet namelijk de commits kunnen ophalen via deze version control tool.
        
        Een build scheduler zorgt er ook voor dat het mogelijk is om de veranderingen aan de code te testen op een test server, indien deze aanwezig is. De build scheduler moet het mogelijk maken om de code te deployen naar de test server waar een realistische opstelling nagemaakt is om zo de code te testen.
        De build scheduler heeft ook als belangrijke taak goede feedback te bezorgen aan de developers indien testen niet slagen. Hoe beter en sneller de feedback, hoe sneller het probleem opgelost kan worden. De developer verliest op deze manier weinig tijd en weet exact waar de code fout gelopen is.
        
        
        
        \subsection{Requirements}
        Nu we weten hoe een build scheduler werkt is het belangrijk om te weten naar welke criteria er moet gekeken worden om de beste tool eruit te kiezen. De functionele en niet-functionele requirements worden opgelijst om een beter beeld te krijgen hoe we een vergelijking moeten toepassen.

            \paragraph{Functionele requirements}
            Een functionele requirement is een functie wat het systeem moet doen.
            De punten die in deze sectie worden aangehaald komen uit hoofdstuk \ref{ch:cicd-pipeline}. Om een duidelijk beeld te krijgen wat de build scheduler moet doen worden de functionele requirements hier nog eens kort besproken.
            \begin{itemize}
                \item De build scheduler moet de aanpassingen aan de code sneller naar de klant brengen
                \item De downtime van een applicatie moet dalen
                \item Er moet vermeden worden dat een applicatie stopt met werken wanneer een fout in de code wordt gedeployed
            \end{itemize}
            
            \paragraph{Niet-functionele requirements}
            Een niet-functionele requirement legt uit hoe het systeem een bepaalde functie moet uitvoeren. Voor Amista is het belangrijk dat de build scheduler voldoet aan hoge security eisen. Vaak wordt er met hele grote projecten gewerkt die gevoelige data en broncode beschikken. Om optimale veiligheid te garanderen wordt gewerkt met een build scheduler die op een on-premise systeem zal draaien. Een andere mogelijkheid, waar vandaag de dag veel in wordt geïnvesteerd, is een cloud build scheduler. Dit zorgt ervoor dat de code buiten de onderneming gaat, wat toch een zeker risico met zich meebrengt.

        \subsection{Criteria}
            \paragraph{Must-haves}
            De must-haves van de build scheduler zullen vooral te maken hebben met de omgeving waarin ze moeten werken. Zoals in de literatuurstudie al beschreven staat, moet een build scheduler gebruikt worden met een SAPUI5 applicatie, samen met een SAP HANA database gehost op SAP Cloud Platform.
            
            In deze thesis wordt dezelfde source control manager gebruikt als bij Amista. Het is niet van toepassing om zelf een source control system en repository manager te kiezen, er moet gebruik gemaakt worden van Git en Bitbucket. Het is vanzelfsprekend dat de build scheduler moet kunnen integreren met Git en Bitbucket.

            Om een betere vergelijking te kunnen maken heeft Amista een Ubuntu server voorzien voor de uitwerking van de proof-of-concept, zie hoofdstuk \ref{ch:proof-of-concept}. Deze dient ook als simulatie voor de realistische on-premise set-up. Het is dus noodzakelijk dat de build scheduler op een Unix-based systeem kan draaien.
            
            De build scheduler moet ook deel kunnen uitmaken van een Continuous Delivery pipeline en automated tests kunnen uitvoeren. Een belangrijk onderdeel van de automated tests binnen SAPUI5 zijn OPA en QUnit  tests, deze moeten zeker uitgevoerd kunnen worden door de build scheduler.
            Binnen SAPUI5 wordt er vaak gebruik gemaakt van %TODO: Karma om bovenstaande tests te automatiseren, daarom is het een must-have dat de build-scheduler de werking van Karma ondersteunt.
            
            Het is voor Amista heel belangrijk dat de build scheduler makkelijk in gebruik is. Ondanks dat deze technologieën vaak nieuw zijn voor de developers zou het leerproces niet lang mogen duren. Ze redeneren dan ook op volgende manier: ze spenderen liever wat meer geld aan een build scheduler waar de developers snel mee weg zijn, dan een goedkopere waar de softwareontwikkelaars een hele tijd over doen om het proces onder de knie te krijgen. Het geld dat ze spenderen aan de duurdere, maar makkelijkere build scheduler is kleiner dan de lonen die ze moeten uitbetalen aan de programmeurs om de tool onder de knie te krijgen.
            
            De tools die we gaan vergelijken moeten ook getest worden op de mogelijkheid tot een audit. Dit omdat Amista ook vaak voor voedingsbedrijven werkt en dit wettelijk verplicht is.
            Veiligheid is ook een belangrijk punt voor Amista, omdat ze heel vaak met zeer gevoelige data van hun klanten werken. Daarom is het belangrijk dat de tools goed omgaan met security. Hoe valt dit te testen? Als de tool meerdere malen per maand een nieuwe versie van de software uitbrengt zijn ze begaan met de veiligheid.
            
            Als we bovenstaande punten even kort samenvatten moet de build scheduler aan volgende vereisten voldoen:
            \begin{itemize}
                \item Hij moet bruikbaar zijn met een SAPUI5 applicatie, een SAP HANA database gehost op SAP Cloud Platform
                \item Hij moet kunnen integreren met Git \& Bitbucket
                \item De build scheduler moet op een Unix-based server kunnen draaien
                \item De tool moet deel uitmaken van een CI/CD pipeline en automated tests kunnen uitvoeren, specifiek de Qunit en OPA tests uit de SAPUI5 applicatie door het gebruik van %TODO: Karma te ondersteunen.
                \item Gebruiksgemak moet centraal staan
                \item Het moet mogelijk zijn om auditing toe te passen
                \item Security is belangrijk, zijn er maandelijks meerdere releases ter beschikking?
            \end{itemize}
            
            
            \paragraph{Should-Haves}
            De build scheduler zou moeten beschikken over gratis gebruik van de software om te experimenteren. Het is de bedoeling dat we in het volgende hoofdstuk een proof-of-concept kunnen uitwerken om bepaalde build schedulers te installeren. Daarom is het nodig om een gratis versie te hebben die de opstelling mogelijk maakt.\\
            Amista communiceert vaak via Skype For Business, ze zouden het leuk vinden om via dit kanaal ook feedback te krijgen over de staat van de build.
            
            
            \paragraph{Nice-to-Haves}
            De build scheduler moet niet per se op een cloud draaien, het belangrijkste is dat de data lokaal gehouden kan worden door de build scheduler op een on-premise installatie op te zetten. Maar Amista wil graag deze optie openhouden voor de toekomst. Daarom is het mooi meegenomen als de gekozen build scheduler aan deze vereiste voldoet, maar het is geen breekpunt.\\
            Bij Amista denken ze aan de toekomst. Het zou leuk zijn als de tool kan integreren met Docker om de builds te bouwen. Omdat dit toch een handige, opkomende tool is binnen de informatica.\\
            Het is geen noodzaak om de build scheduler informatie via Slack of WhatsApp te versturen, maar een pluspunt.\\
            De CTO van Amista zou het ook als een pluspunt zien als de build scheduler het mogelijk maakt om een rapport te genereren wat er de voorbije week/maand gebeurd is in de code. Wie heeft wat gedaan, wie maakt de meeste fouten, ...? Zo'n zaken zijn handig om de prestaties van de developers binnen een team te kennen en te versterken.

        \subsection{Long list}
        Nu de verschillende criteria gekend zijn kan de effectieve vergelijking tussen de build schedulers gebeuren.
        Eerst zal er een korte uitleg gegeven worden over welke programma's we gaan vergelijken en nadien vindt de vergelijking plaats in een overzichtelijke tabel.

            %TODO: https://www.altexsoft.com/blog/engineering/comparison-of-most-popular-continuous-integration-tools-jenkins-teamcity-bamboo-travis-ci-and-more/
            
            \paragraph{Jenkins}
            Jenkins is de meest gekende build scheduler binnen de IT-wereld. Het is een op zichzelf staande, open source automation server dat ontstaan is in 2011 na een afscheiding van het Hudson project. Jenkins is vooral bekend om de vele plug-ins die het mogelijk maken om met bijna alle talen en platformen te integreren. Jenkins is geschreven in Java en draait op een Java platform.
            
            %Links gevonden voor tabel: 
            %\begin{itemize}
            %    \item https://sap.github.io/jenkins-library/scenarios/ui5-sap-cp/Readme/
            %    \item https://qxf2.com/blog/blog-post-on-how-to-connect-bitbucket-with-jenkins/
            %    \item https://www.digitalocean.com/community/tutorials/how-to-install-jenkins-on-ubuntu-18-04
            %    \item https://wiki.jenkins.io/display/JENKINS/Delivery+Pipeline+Plugin
            %    \item Auditing: https://wiki.jenkins.io/display/JENKINS/Audit+Trail+Plugin
            %\end{itemize}
            
            \paragraph{Circle CI}
            Circle CI is opgericht in 2011 in San Francisco en wordt door vele grote bedrijven gebruikt in hun Continuous Integration pipeline. Het grootste deel van Circle CI is geschreven in Clojure en de Frontend in ClojureScript. 

            \paragraph{Bamboo}
            Bamboo, een Java-based Continuous Integration en Continuous Delivery tool, werd opgericht in 2007 door het bedrijf Atlassian dat ook Bitbucket onder zijn hoede heeft.
            
            \paragraph{Travis CI}
            Travis CI is een integration tool dat geschreven is in Ruby en opgericht in 2011 in Duitsland. Het staat bekend om zijn samenwerking met GitHub.
            
            
            %TODO: https://www.tablesgenerator.com
            %TODO: Kijken voor de integratie met Karma ipv te vergelijken op het ondersteunen van OPA tests
            %TODO: Kijken of Travis enkel met GitHub kan samenwerken en ook niet bedoeld is om continuous delivery toe te passen
            \begin{landscape}
                \begin{table}[]
                    \centering
                    \begin{tabular}{c|c|cccccc}
                        \cline{2-2}
                        \textbf{} & \textbf{Must-have} & \textbf{} & \textbf{} & \textbf{} & \textbf{} & \textbf{} & \textbf{} \\ \cline{2-8} 
                        & \textbf{SAP Cloud Platform \& UI5} & \multicolumn{1}{c|}{\textbf{Git}} & \multicolumn{1}{c|}{\textbf{Bitbucket}} & \multicolumn{1}{c|}{\textbf{On premise}} & \multicolumn{1}{c|}{\textbf{CI/CD pipeline}} & \multicolumn{1}{c|}{\textbf{Automated tests}} & \multicolumn{1}{c|}{\textbf{Karma}} \\ \hline
                        \multicolumn{1}{|c|}{\textbf{Jenkins}} & Ja & \multicolumn{1}{c|}{Ja} & \multicolumn{1}{c|}{Ja} & \multicolumn{1}{c|}{Ja} & \multicolumn{1}{c|}{Ja} & \multicolumn{1}{c|}{Ja} & \multicolumn{1}{c|}{INVULLEN} \\ \hline
                        \multicolumn{1}{|c|}{\textbf{Circle CI}} & Geen info & \multicolumn{1}{c|}{Ja} & \multicolumn{1}{c|}{Ja} & \multicolumn{1}{c|}{Ja} & \multicolumn{1}{c|}{Ja} & \multicolumn{1}{c|}{Ja} & \multicolumn{1}{c|}{INVULLEN} \\ \hline
                        \multicolumn{1}{|c|}{\textbf{Bamboo}} & 1 blogpost & \multicolumn{1}{c|}{Ja} & \multicolumn{1}{c|}{Ja} & \multicolumn{1}{c|}{Ja} & \multicolumn{1}{c|}{Ja} & \multicolumn{1}{c|}{Ja} & \multicolumn{1}{c|}{INVULLEN} \\ \hline
                        \multicolumn{1}{|c|}{\textbf{Travis CI}} & Ja & \multicolumn{1}{c|}{Ja} & \multicolumn{1}{c|}{Nee, enkel GitHub} & \multicolumn{1}{c|}{Ja} & \multicolumn{1}{c|}{Ja} & \multicolumn{1}{c|}{Ja} & \multicolumn{1}{c|}{INVULLEN} \\ \hline
                    \end{tabular}
                    \caption[Long list CI/CD tools, nummer 1]{Long list van vergelijking CI/CD tools, nummer 1}
                    \label{tab:long-list1}
                \end{table}
            \end{landscape}
        
            \begin{landscape}
                \begin{table}[]
                    \centering
                    \begin{tabular}{c|c|cc|c|l}
                        \cline{2-2} \cline{5-5}
                        \textbf{} & \textbf{Must-have} & \textbf{} & \textbf{} & \textbf{Should-haves} &  \\ \cline{2-6} 
                        & \textbf{Gebruiksgemak} & \multicolumn{1}{c|}{\textbf{Auditing}} & \textbf{Meerdere releases / maand} & \textbf{Prijs} & \multicolumn{1}{l|}{\textbf{Trial}} \\ \hline
                        \multicolumn{1}{|c|}{\textbf{Jenkins}} & Makkelijk & \multicolumn{1}{c|}{Ja} & Ja & Gratis & \multicolumn{1}{l|}{Gratis} \\ \hline
                        \multicolumn{1}{|c|}{\textbf{Circle CI}} & Makkelijk & \multicolumn{1}{c|}{Basic} & Meestal & \$35 / user & \multicolumn{1}{l|}{20 dagen} \\ \hline
                        \multicolumn{1}{|c|}{\textbf{Bamboo}} & Middelmatig & \multicolumn{1}{c|}{Ja} & Nee & \begin{tabular}[c]{@{}c@{}}\$1100 / jaar\\ voor 1 remote agent\\ en ongelimiteerd aantal jobs\end{tabular} & \multicolumn{1}{l|}{30 dagen} \\ \hline
                        \multicolumn{1}{|c|}{\textbf{Travis CI}} & Makkelijk & \multicolumn{1}{c|}{Ja} & Ja & \begin{tabular}[c]{@{}c@{}}\$2739 / jaar\\ voor 5 jobs\end{tabular} & \multicolumn{1}{l|}{\begin{tabular}[c]{@{}l@{}}Gratis voor\\ open source\end{tabular}} \\ \hline
                    \end{tabular}
                    \caption[Long list CI/CD tools, nummer 2]{Long list van vergelijking CI/CD tools, nummer 2}
                    \label{tab:long-list2}
                \end{table}
            \end{landscape}
        
            \begin{landscape}
                \begin{table}[]
                    \centering
                    \begin{tabular}{c|c|c|c|c}
                        \cline{2-2} \cline{4-4}
                        \textbf{} & \textbf{Should-haves} & \textbf{} & \textbf{Nice-to-haves} & \textbf{} \\ \cline{2-5} 
                        & \textbf{Skype for Business} & \textbf{Container support} & \textbf{Slack} & \multicolumn{1}{c|}{\textbf{Rapport}} \\ \hline
                        \multicolumn{1}{|c|}{\textbf{Jenkins}} & Ja & Ja & Ja & \multicolumn{1}{c|}{Ja} \\ \hline
                        \multicolumn{1}{|c|}{\textbf{Circle CI}} & Nee & Ja & Ja & \multicolumn{1}{c|}{Ja} \\ \hline
                        \multicolumn{1}{|c|}{\textbf{Bamboo}} & Nee & Ja & \begin{tabular}[c]{@{}c@{}}Ja,\\ via derde partij\end{tabular} & \multicolumn{1}{c|}{Nee} \\ \hline
                        \multicolumn{1}{|c|}{\textbf{Travis CI}} & Ja & Ja & Ja & \multicolumn{1}{c|}{Ja} \\ \hline
                    \end{tabular}
                    \caption[Long list CI/CD tools, nummer 3]{Long list van vergelijking CI/CD tools, nummer 3}
                    \label{tab:long-list3}
                \end{table}
            \end{landscape}


    \section{Voorbeeldapplicatie dat Amista zal gebruiken}
    Hoe ziet de omgeving eruit waar Amista een CI/CD pipeline in wil opzetten? In de literatuurstudie worden de gebruikte tools binnen SAP uitgelegd, in dit hoofdstuk worden de onderliggende relaties tussen de tools besproken.
    Amista heeft een Ubuntu server ter beschikking gesteld om te experimenteren. Voor deze thesis wordt de server gebruikt als Continuous Integration server, zo kunnen de build schedulers op de beste manier vergeleken worden zonder veel externe factoren.
    
        \subsection{Build scheduler server}
        Amista heeft een Ubuntu server ter beschikking gesteld dat wordt gehost op Digital Ocean, een Amerikaanse hosting bedrijf. Digital Ocean was de derde grootse speler op de markt wanneer men spreekt over web-hosting computers. Ze zijn gespecialiseerd in cloud based oplossingen en hebben datacenters over heel de wereld verspreid. De server wordt gebruikt om de build schedulers op te draaien en zo te vergelijken. Eerst moeten er enkele belangrijke zaken ingesteld worden alvorens aan de slag te gaan, zoals security en dergelijke.
            
            \paragraph{Ubuntu server}
            Ubuntu is een Linuxdistributie, gebaseerd op het bekende Debian en gekend is vanwege de open-source eigenschappen.
            Deze versie van Linux wordt vooral gebruikt voor cloud computing, vandaar dat Digital Ocean voor kiest om met een Ubuntu 18.04 te werken.
            De versie 18.04 wordt ook wel 'Bionic Beaver' genoemd.
            Amista heeft voor de uitwerking van de proof-of-concept\ref{ch:proof-of-concept} een server voorzien met een x64-bit Operating System waar 1 CPU en 1GB RAM ter beschikking staat.
            
            \paragraph{Installatie Ubuntu server}
            Eerst maken we de Ubuntu server klaar voor gebruik om zo de nodige zaken te installeren.
            In figuur \ref{OpzettenServer1} in de bijlagen wordt er getoond wat er moet gebeuren als er voor de eerste keer ingelogd wordt op de server.
            Via de root account wordt er via SSH ingelogd op de server. 
            
            \begin{figure}
                \centering
                \includegraphics[scale=0.7]{OpzettenServer1}
                \caption{Opzetten Server, stap 1} \label{OpzettenServer1}
            \end{figure}
            
            SSH staat voor secure shell en is een software protocol dat voor een veilige verbinding (tunnel) zorgt tussen de client en de server. Het wordt gebruikt voor het configureren van een server, het beheren van netwerken en operating systems. Alle gegevens dat tussen beiden worden uitgevoerd zijn geëncrypteerd waardoor het moeilijker wordt voor hackers om de data te bemachtigen.
            
            Een server die gebruik maakt van SSH wordt ook wel een sshd server genoemd. Eens ingelogd op de sshd server moeten er enkele zaken aangepast worden aan de ssh configuratie in de file '/etc/ssh/sshd\_config'. Omdat we hier via de root gebruiker werken moet de property PermitRootLogin op yes staan. Dit zorgt ervoor dat de root gebruiker kan inloggen.
            StrictMode moet ook op yes staan, zo kan er niemand inloggen als de authenticatie documenten leesbaar zijn voor iedereen. Dit voor het beveiligen van configuratie documenten. Deze configuratie kan u zien in figuur\ref{OpzettenServer2} hier onder.
            
            \begin{figure}
                \centering
                \includegraphics[scale=0.75]{OpzettenServer2}
                \caption{Opzetten Server, stap 2} \label{OpzettenServer2}
            \end{figure}
            
            De default manier om in te loggen via ssh is via een account en een paswoord, maar het is ook mogelijk om het account en het paswoord te vervangen door een private en een public key. Dit principe noemt key-based authentication en wordt vooral tijdens development en in scripts gebruikt of voor single sign-on. SSH genereert een private en een public key op de client wanneer deze stap wordt geconfigureerd. De private key moet veilig bewaard worden op de client computer. De public key moet doorgegeven worden aan de remote server. Wanneer de client wil inloggen op de server voert hij een request uit. De server maakt via zijn public key een bericht en stuurt dit als response door naar de client. De client leest het bericht aan de hand van zijn private key en stuurt dan een aangepaste response terug naar de remote server. De server valideert deze response. Bij een geldige private key zal er een goede response verstuurd worden, bij een ongeldige private key een foute response.
            In deze thesis gaat men ervan uit dat de client computer een ssh key heeft die gebruikt kan worden. Zoals u in figuur\ref{OpzettenServer3} kan zien heeft de client die gebruikt werd tijdens het schrijven van deze thesis enkel lees- en schrijfrechten voor de file 'id\_rsa', dit om het als secret te bewaren.
            De 'id\_rsa.pub' is de publieke key van de client die op de server moet komen om zo de ssh validatie te voorzien, dit wordt ook wel een ssh session genoemd. Eens de ssh session geconfigureerd is zal het niet nodig zijn om via een paswoord in te loggen op de remote server via deze client.
            In figuur\ref{OpzettenServer4} in de bijlagen is te zien hoe de ssh session wordt opgezet tussen de client en de remote server voor de root user.
            Voor deze thesis en om veiligheidsredenen is het beter om enkel via key-based authenticatie in te loggen en het paswoord uit te sluiten.
            In figuur\ref{OpzettenServer5} is te zien welke aanpassingen in de sshd\_config file van de sshd server moeten gebeuren om het niet meer mogelijk te maken om in te loggen via een paswoord. PasswordAuthentication en ChallengeResponseAuthentication moeten naar no verandert worden. PubKeyAuthentication moet naar yes verandert worden.
            Nu moet de 'sshd\_config' file opgeslagen worden (\^x + y + enter) en de ssh daemon herstart worden door het commando 'sudo systemctl restart ssh' in te geven.
            
            \begin{figure}
                \centering
                \includegraphics{OpzettenServer3}
                \caption{Opzetten Server, stap 3} \label{OpzettenServer3}
            \end{figure}
        
            \begin{figure}
                \centering
                \includegraphics[scale=0.65]{OpzettenServer4}
                \caption{Opzetten Server, stap 4} \label{OpzettenServer4}
            \end{figure}
        
            \begin{figure}
                \centering
                \includegraphics[scale=0.75]{OpzettenServer5}
                \caption{Opzetten Server, stap 5} \label{OpzettenServer5}
            \end{figure}
            
            Om de remote server nog meer te beschermen tegen cyber aanvallen is het nodig om een firewall op te zetten. In deze voorbeeldapplicatie maken we gebruik van de UFW Firewall. Dis staat voor Uncomplicated Firewall en is een gebruiksvriendelijke tool dat helpt om de iptables onder controle te houden om zo te zorgen dat bepaalde services toegelaten worden tot onze server.
            In Linux maken ze gebruik van het protocol SSH via de service OpenSSH, deze heeft ook een profiel bij UFW.
            In Figuur\ref{OpzettenServer6} in de bijlagen is te zien hoe de firewall de SSH service toelaat. Het is enkel mogelijk om de server te bereiken via deze service. Later worden er uiteraard meerdere services toegelaten.
            
            \begin{figure}
                \centering
                \includegraphics[scale=0.9]{OpzettenServer6}
                \caption{Opzetten Server, stap 6} \label{OpzettenServer6}
            \end{figure}
            
            Nu alle stappen voor de configuratie van de server gedaan zijn is het zeer makkelijk om in te loggen op de server.
            Het is hetzelfde als de eerste keer, maar nu vraagt de server niet meer naar een paswoord, maar gebruikt hij de ssh-key. Het is voldoende om 
            'ssh root@188.166.61.128' te typen om in te loggen.
            Als je wil uitloggen is het nodig om in de command line van de server exit te typen.
    
        \subsection{Database}
        Binnen SAP wordt een HANA database aangeraden om te gebruiken. Momenteel is versie 2.0 van SAP HANA op de markt en deze versie biedt tal van extra mogelijkheden ten opzichte van de vorige versie. SAP HANA wordt zeer goed ondersteund door de andere programma's binnen SAP en wordt daarom ook wel veel gebruikt. Maar het enige nadeel is dat deze nog geen ondersteuning biedt voor het gebruik als destination in SAP Cloud Platform. Dit maakt het praktisch niet haalbaar om deze database aan te spreken in een SAPUI5 applicatie op SAP Cloud Platform.
        Daarom is er samen met Pieter-Jan beslist om toch voor HANA 1.0 te kiezen om de proof-of-concept uit te werken. De principes blijven hetzelfde, alleen is er zekerheid dat deze versie gebruikt kan worden als destination.
        
        Zoals eerder al aangegeven is een Source Code Repository van groot belang voor een CI/CD pipeline en development in het algemeen, daarom is het belangrijk dat dit als eerste stap gebeurt alvorens de database aan te maken.
        
            \paragraph{Source Control \& Databank module}
            In figuur\ref{SourceControl1} kan u waarnemen hoe een repository gemaakt wordt in Bitbucket.
            Eens de repository aangemaakt is heb je het webadres nodig om de clone te maken op je lokale machine.
            
            \begin{figure}
                \centering
                \includegraphics[scale=0.5]{SourceControl1}
                \caption{Source Control, stap 1} \label{SourceControl1}
            \end{figure}
            
            De volgende stap is het project aanmaken. Dit doen we door een SAP Cloud Platform trial account user aan te maken, dit is een gratis dienst van SAP met beperkt mogelijkheden. Wat wel tot de mogelijkheden behoort is het maken van een HANA database en een SAPUI5 webapplicatie, ideaal voor deze bachelorproef. Eens een account aangemaakt moet u naar de Neo Trial service gaan die verbonden is met deze account. Daar ziet u in de linkse kolom 'SAP HANA / SAP ASE' met daaronder 'Databases \& Schemas' staan zoals in figuur \ref{HANA1} te zien is. 
            Nu is het nodig om op 'New' te klikken. Er verschijnt een dialog waar de gegevens, zoals te zien is in figuur \ref{HANA2}, ingevuld moeten worden. Eens de juiste gegevens ingevuld zijn moet er op 'Create' gedrukt worden. Na enkele minuten kan je via het tabblad 'Overview' naar de SAP HANA Cockpit gaan. Klik op de link en log in met user 'System' en het wachtwoord dat je net ingesteld hebt. Er worden automatisch enkele permissies gegeven en je krijgt de rol als Administrator toegekend. Deze stappen kan je volgen in figuren \ref{HANA3} en \ref{HANA4}.
            
            Nu moet er terug gegaan worden naar het tabblad zoals in figuur \ref{HANA3}. Daar staat de link naar de SAP HANA Workbench waar je op moet klikken. Eens het nieuwe venster geopend is moet je op 'Security' klikken en in de linkse kolom de Users openen, zie figuren \ref{HANA5} en \ref{HANA6}. Er moeten eerst bepaalde rollen aan de SYSTEM-user toegekend worden alvorens zelf een user te creëren zoals te zien valt in figuur \ref{HANA7}.\\
            Nu is de stap aangebroken om een nieuwe user aan te maken, geef die jouw voornaam en voeg de rollen toe zoals in de figuren te zien is. Er moet ook een package toegevoegd worden zoals te volgen valt in figuren \ref{HANA8}, \ref{HANA9}, \ref{HANA10} en \ref{HANA11}.
            
            Keer terug naar het tabblad zoals in figuur \ref{HANA5} en klik op 'Catalog' en maak een nieuw schema, 'DEMO' genaamd aan. 
            Keer terug naar het tabblad van 'Security' en volg de stappen zoals in figuren \ref{HANA12}, \ref{HANA13} en \ref{HANA14}.
            Nu moet je uitloggen als System-user en inloggen met je eigen account. Als je de eerste maal inlogt met je eigen account vraagt hij om je een nieuw wachtwoord in te stellen. Eens je ingelogd bent met je eigen user moet je teruggaan naar het tabblad zoals in figuur \ref{HANA5} en op 'Editor' klikken. De volgende stap is een nieuw package, 'DEMO' genaamd toevoegen. In dit package moet er nog eentje komen, dat 'DATA' moet heten. Daarin moet er een nieuwe file 'data.hdbdd' gemaakt worden met data. Deze stappen kan u volgen in figuren \ref{HANA15}, \ref{HANA16}, \ref{HANA17}, \ref{HANA18}, \ref{HANA19}, \ref{HANA20} en \ref{HANA21}.\\
            Een realistische opstelling van een CI/CD voorbeeldapplicatie start bijna nooit vanaf nul, het bouwt meestal voort op bestaande, geschreven software.
            Daarom zal er in deze voorbeeldapplicatie manueel een basis gelegd worden, zo kan er aan de hand van een build scheduler voortgebouwd worden op deze geschreven software.
            De bescheiden database waar we naartoe willen gaan bestaat uit slechts één entiteit: een Artiest die volgende properties heeft: ID, Naam en JaarVanOorsprong.
      
            \begin{figure}
                \centering
                \includegraphics[scale=0.25]{HANA1}
                \caption{Opzetten HANA, stap 1} \label{HANA1}
            \end{figure}
        
            \begin{figure}	
                \centering
                \includegraphics[scale=0.35]{HANA2}
                \caption{Opzetten HANA, stap 2} \label{HANA2}
            \end{figure}
            
            \begin{figure}	
                \centering
                \includegraphics[scale=0.35]{HANA3}
                \caption{Opzetten HANA, stap 3} \label{HANA3}
            \end{figure}
        
            \begin{figure}	
                \centering
                \includegraphics[scale=0.35]{HANA4}
                \caption{Opzetten HANA, stap 4} \label{HANA4}
            \end{figure}
        
            \begin{figure}	
                \centering
                \includegraphics[scale=0.35]{HANA5}
                \caption{Opzetten HANA, stap 5} \label{HANA5}
            \end{figure}
        
            \begin{figure}	
                \centering
                \includegraphics[scale=0.2]{HANA6}
                \caption{Opzetten HANA, stap 6} \label{HANA6}
            \end{figure}
        
            \begin{figure}	
                \centering
                \includegraphics[scale=0.35]{HANA7}
                \caption{Opzetten HANA, stap 7} \label{HANA7}
            \end{figure}
            
            \begin{figure}	
                \centering
                \includegraphics[scale=0.35]{HANA8}
                \caption{Opzetten HANA, stap 8} \label{HANA8}
            \end{figure}
            
            \begin{figure}	
                \centering
                \includegraphics[scale=0.35]{HANA9}
                \caption{Opzetten HANA, stap 9} \label{HANA9}
            \end{figure}
            
            \begin{figure}	
                \centering
                \includegraphics[scale=0.35]{HANA10}
                \caption{Opzetten HANA, stap 10} \label{HANA10}
            \end{figure}
            
            \begin{figure}	
                \centering
                \includegraphics[scale=0.35]{HANA11}
                \caption{Opzetten HANA, stap 11} \label{HANA11}
            \end{figure}
            
            \begin{figure}	
                \centering
                \includegraphics[scale=0.35]{HANA12}
                \caption{Opzetten HANA, stap 12} \label{HANA12}
            \end{figure}
            
            \begin{figure}	
                \centering
                \includegraphics[scale=0.35]{HANA13}
                \caption{Opzetten HANA, stap 13} \label{HANA13}
            \end{figure}
            
            \begin{figure}	
                \centering
                \includegraphics[scale=0.35]{HANA14}
                \caption{Opzetten HANA, stap 14} \label{HANA14}
            \end{figure}
            
            \begin{figure}	
                \centering
                \includegraphics[scale=0.35]{HANA15}
                \caption{Opzetten HANA, stap 15} \label{HANA15}
            \end{figure}
            
            \begin{figure}	
                \centering
                \includegraphics[scale=0.35]{HANA16}
                \caption{Opzetten HANA, stap 16} \label{HANA16}
            \end{figure}
            
            \begin{figure}	
                \centering
                \includegraphics[scale=0.35]{HANA17}
                \caption{Opzetten HANA, stap 17} \label{HANA17}
            \end{figure}
            
            \begin{figure}	
                \centering
                \includegraphics[scale=0.35]{HANA18}
                \caption{Opzetten HANA, stap 18} \label{HANA18}
            \end{figure}
            
            \begin{figure}	
                \centering
                \includegraphics[scale=0.35]{HANA19}
                \caption{Opzetten HANA, stap 19} \label{HANA19}
            \end{figure}
            
            \begin{figure}	
                \centering
                \includegraphics[scale=0.35]{HANA20}
                \caption{Opzetten HANA, stap 20} \label{HANA20}
            \end{figure}
            
            \begin{figure}	
                \centering
                \includegraphics[scale=0.35]{HANA21}
                \caption{Opzetten HANA, stap 21} \label{HANA21}
            \end{figure}
        
        
            Nu is de stap aangebroken om met Eclipse te werken. Download Eclipse Neon en installeer het programma.
            Eens geopend moet je de HANA Eclipse Plugin downloaden door in de linkse bovenhoek op 'Help' te klikken en dan naar 'Install New Software...' te gaan. In het invulveld typ je volgende url: 'https://tools.hana.ondemand.com/neon/' en je selecteert het package met naam 'SAP HANA Tools', druk op 'Next', accepteer de voorwaarden en herstart Eclipse.\\
            Sluit het welcome-venster en open een SAP HANA Development door op het teken van 'Open Perspective' te klikken en dan 'SAP HANA Development' te kiezen. Onder het tabblad 'Systems' moet er een nieuw 'Cloud System' toegevoegd worden. De gegevens uit de figuur moet ingevoerd worden met als Subaccount name uw p-nummer dat je bij het initialiseren van je trial account gekregen hebt. Hier moet het woord trial achter geplakt worden. Bij User name volstaat het om enkel met je p-nummer in te geven. Wanneer op 'next' geklikt wordt verschijnt er een nieuwe dialog om te verbinden met een SAP HANA Schema and Database. Vul de gegevens in zoals in de figuur \ref{HANA30}. De voorgaande stappen kan u terugvinden in figuren \ref{HANA22}, \ref{HANA23}, \ref{HANA24}, \ref{HANA25}, \ref{HANA26}, \ref{HANA27}, \ref{HANA28} en \ref{HANA29}.
            Wanneer u de voorgaande stappen correct hebt uitgevoerd verschijnt een soortgelijk scherm zoals in figuur \ref{HANA31}.
            
            Er moet natuurlijk ook data in de database zitten om deze te kunnen gebruiken in de SAPUI5 applicatie. Een manier om dit te doen is via het uploaden van csv-bestanden waar de data gescheiden is door een komma. Maak eerst een csv-bestand aan en vul het met data zoals in figuur \ref{HANA32}. Volg dan de stappen zoals in onderstaande figuren \ref{HANA33}, \ref{HANA34}, \ref{HANA35}, \ref{HANA36}, \ref{HANA37}, \ref{HANA38} en \ref{HANA39} beschreven om de data te importeren.
         
            
            \begin{figure}	
                \centering
                \includegraphics[scale=0.35]{HANA22}
                \caption{Opzetten HANA, stap 22} \label{HANA22}
            \end{figure}
            
            \begin{figure}	
                \centering
                \includegraphics[scale=0.35]{HANA23}
                \caption{Opzetten HANA, stap 23} \label{HANA23}
            \end{figure}
        
            \begin{figure}	
                \centering
                \includegraphics[scale=0.35]{HANA24}
                \caption{Opzetten HANA, stap 24} \label{HANA24}
            \end{figure}
            
            \begin{figure}	
                \centering
                \includegraphics[scale=0.35]{HANA25}
                \caption{Opzetten HANA, stap 25} \label{HANA25}
            \end{figure}
            
            \begin{figure}	
                \centering
                \includegraphics[scale=0.35]{HANA26}
                \caption{Opzetten HANA, stap 26} \label{HANA26}
            \end{figure}
            
            \begin{figure}	
                \centering
                \includegraphics[scale=0.35]{HANA27}
                \caption{Opzetten HANA, stap 27} \label{HANA27}
            \end{figure}
        
            \begin{figure}	
                \centering
                \includegraphics[scale=0.35]{HANA28}
                \caption{Opzetten HANA, stap 28} \label{HANA28}
            \end{figure}
            
            \begin{figure}	
                \centering
                \includegraphics[scale=0.35]{HANA29}
                \caption{Opzetten HANA, stap 29} \label{HANA29}
            \end{figure}
        
            \begin{figure}	
                \centering
                \includegraphics[scale=0.35]{HANA30}
                \caption{Opzetten HANA, stap 30} \label{HANA30}
            \end{figure}
            
            \begin{figure}	
                \centering
                \includegraphics[scale=0.35]{HANA31}
                \caption{Opzetten HANA, stap 31} \label{HANA31}
            \end{figure}
        
            \begin{figure}	
                \centering
                \includegraphics[scale=0.35]{HANA32}
                \caption{Opzetten HANA, stap 32} \label{HANA32}
            \end{figure}
            
            \begin{figure}	
                \centering
                \includegraphics[scale=0.35]{HANA33}
                \caption{Opzetten HANA, stap 33} \label{HANA33}
            \end{figure}
        
            \begin{figure}	
                \centering
                \includegraphics[scale=0.35]{HANA34}
                \caption{Opzetten HANA, stap 34} \label{HANA34}
            \end{figure}
            
            \begin{figure}	
                \centering
                \includegraphics[scale=0.35]{HANA35}
                \caption{Opzetten HANA, stap 35} \label{HANA35}
            \end{figure}
        
            \begin{figure}	
                \centering
                \includegraphics[scale=0.35]{HANA36}
                \caption{Opzetten HANA, stap 36} \label{HANA36}
            \end{figure}
            
            \begin{figure}	
                \centering
                \includegraphics[scale=0.35]{HANA37}
                \caption{Opzetten HANA, stap 37} \label{HANA37}
            \end{figure}
        
            \begin{figure}	
                \centering
                \includegraphics[scale=0.35]{HANA38}
                \caption{Opzetten HANA, stap 38} \label{HANA38}
            \end{figure}
            
            \begin{figure}	
                \centering
                \includegraphics[scale=0.35]{HANA39}
                \caption{Opzetten HANA, stap 39} \label{HANA39}
            \end{figure}
        
            Eenmaal de data in de database zit is het logisch om deze er ook uit te krijgen. Dit gaan we configureren in volgende stappen.\\
            Ga terug naar de Workbench van HANA Development en creëer een nieuwe package onder DATA genaamd 'SERVICES'. In deze package moet een nieuwe file toegevoegd worden: 'services.xsodata' om de data als OData-service ter beschikking te stellen.\\
            Er moet een applicatie gemaakt worden in de package SERVICES door er met de rechtermuisknop op te klikken en 'Create Application' te kiezen. Vul de gegevens in zoals in de figuur \ref{HANA45} is aangegeven. In de 'services.xsodata' file moeten de gegevens ingevuld worden zoals in figuur \ref{HANA46}. De andere stappen zijn te volgen in figuren \ref{HANA40}, \ref{HANA41}, \ref{HANA42}, \ref{HANA43} en \ref{HANA44}.
            
            Het is mogelijk om vanaf nu de data te bekijken als OData door de 'services.xsodata' file te openen en op de run knop te klikken. Deze opent een url en het is nodig om na de '.com' de url te vervangen door '/DEMO/SERVICES/services.xsodata/\$metadata' om de metadata-file te openen.
            Om de data uit de tabel Artiesten te halen is het nodig om de url na de '.com' te vervangen door '/DEMO/SERVICES/services.xsodata/Artiesten'.\\
            Als deze stap lukt is de OData-service te bereiken als een destination in SAP Cloud Platform. Ga nu naar SAP Cloud Platform om een nieuwe destination toe te voegen. Volg de stappen in figuren \ref{HANA47} en \ref{HANA48}.
            
            \begin{figure}	
                \centering
                \includegraphics[scale=0.35]{HANA40}
                \caption{Opzetten HANA, stap 40} \label{HANA40}
            \end{figure}
            
            \begin{figure}	
                \centering
                \includegraphics[scale=0.35]{HANA41}
                \caption{Opzetten HANA, stap 41} \label{HANA41}
            \end{figure}
            
            \begin{figure}	
                \centering
                \includegraphics[scale=0.35]{HANA42}
                \caption{Opzetten HANA, stap 42} \label{HANA42}
            \end{figure}
            
            \begin{figure}	
                \centering
                \includegraphics[scale=0.35]{HANA43}
                \caption{Opzetten HANA, stap 43} \label{HANA43}
            \end{figure}
            
            \begin{figure}	
                \centering
                \includegraphics[scale=0.35]{HANA44}
                \caption{Opzetten HANA, stap 44} \label{HANA44}
            \end{figure}
        
            \begin{figure}	
                \centering
                \includegraphics[scale=0.35]{HANA45}
                \caption{Opzetten HANA, stap 45} \label{HANA45}
            \end{figure}
            
            \begin{figure}	
                \centering
                \includegraphics[scale=0.35]{HANA46}
                \caption{Opzetten HANA, stap 46} \label{HANA46}
            \end{figure}
            
            \begin{figure}	
                \centering
                \includegraphics[scale=0.35]{HANA47}
                \caption{Opzetten HANA, stap 47} \label{HANA47}
            \end{figure}
            
            \begin{figure}	
                \centering
                \includegraphics[scale=0.35]{HANA48}
                \caption{Opzetten HANA, stap 48} \label{HANA48}
            \end{figure}
                    
        \subsection{UI5-applicatie}
        
        
        \subsection{Automated testing binnen SAPUI5}
        %TODO: https://help.sap.com/viewer/468a97775123488ab3345a0c48cadd8f/7.52.1/en-US/ae448243822448d8ba04b4784f4b09a0.html
        %TODO: https://blogs.sap.com/2016/11/21/headless-opa5-testing-with-karma-and-phantomjs/