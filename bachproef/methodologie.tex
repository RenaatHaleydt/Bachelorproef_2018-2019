%%=============================================================================
%% Methodologie
%%=============================================================================

\chapter{\IfLanguageName{dutch}{Methodologie}{Methodology}}
\label{ch:methodologie}

Deze thesis gaat op zoek naar de beste build scheduler voor de specifieke set-up die ze bij Amista hanteren: een SAPUI5 webapplicatie met een SAP HANA database draaiend via Node.js en alles gehost op SAP Cloud Platform.

%% TODO: Hoe ben je te werk gegaan? Verdeel je onderzoek in grote fasen, en
%% licht in elke fase toe welke stappen je gevolgd hebt. Verantwoord waarom je
%% op deze manier te werk gegaan bent. Je moet kunnen aantonen dat je de best
%% mogelijke manier toegepast hebt om een antwoord te vinden op de
%% onderzoeksvraag.


\section{Voorbeeldapplicatie dat Amista zal gebruiken}
\label{sec:voorbeeldapplicatie}
Hoe ziet de omgeving eruit waar een CI/CD pipeline geïntegreerd moet worden? In de literatuurstudie is de uitleg over de tools terug te vinden, hier worden de onderliggende relaties tussen de tools besproken.
Amista heeft een Ubuntu server ter beschikking gesteld om te experimenteren. Ze geven carte blanche wat er met de server moet gebeuren.
Voor deze thesis wordt de server gebruikt als Continuous Integration server. Zo kunnen de build schedulers op de beste manier vergeleken worden zonder veel externe factoren.

\section{Vergelijking van de build schedulers}
\label{sec:Vergelijking-build-schedulers}
Wat zijn voor Amista nu de beste build schedulers om een CI/CD pipeline te integreren in hun software development? Welke tools scoren het beste op vlak van snelheid, configureerbaarheid met SAP, geheugenverbruik en de tools die Amista gebruikt en documentatie die te vinden is online.

\section{Handleiding}
\label{sec:handleiding}