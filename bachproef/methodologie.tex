%%=============================================================================
%% Methodologie
%%=============================================================================

\chapter{\IfLanguageName{dutch}{Methodologie}{Methodology}}
\label{ch:methodologie}

%% TODO: Hoe ben je te werk gegaan? Verdeel je onderzoek in grote fasen, en
%% licht in elke fase toe welke stappen je gevolgd hebt. Verantwoord waarom je
%% op deze manier te werk gegaan bent. Je moet kunnen aantonen dat je de best
%% mogelijke manier toegepast hebt om een antwoord te vinden op de
%% onderzoeksvraag.


\section{Onderzoeksdomein 1: Voorbeeldapplicatie dat Amista zal gebruiken}
\label{sec:onderzoeksdeel1}
Hoe ziet de omgeving er uit waar een CI/CD pipeline geïntegreerd moet worden er uit? In de literatuurstudie is de uitleg over de tools terug te vinden, maar hier worden de onderliggende relaties tussen de tools besproken.
Er zal een test omgeving opgezet worden die hier besproken zal worden. 


\section{Onderzoeksdomein 2: Vergelijken van de beschikbare tools}
\label{sec:onderzoeksdeel2}
Wat zijn voor Amista nu de beste tools om een CI/CD pipeline te integreren in hun software development? Welke tools scoren het beste op vlak van snelheid, configureerbaarheid met SAP en de tools die Amista gebruikt en documentatie die te vinden is online.