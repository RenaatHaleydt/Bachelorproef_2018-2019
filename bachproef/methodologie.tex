%%=============================================================================
%% Methodologie
%%=============================================================================

\chapter{\IfLanguageName{dutch}{Methodologie}{Methodology}}
\label{ch:methodologie}

%% TODO: Hoe ben je te werk gegaan? Verdeel je onderzoek in grote fasen, en
%% licht in elke fase toe welke stappen je gevolgd hebt. Verantwoord waarom je
%% op deze manier te werk gegaan bent. Je moet kunnen aantonen dat je de best
%% mogelijke manier toegepast hebt om een antwoord te vinden op de
%% onderzoeksvraag.

\section{Onderzoeksdomein 1: Wat zijn de voor- en nadelen van een CI/CD pipeline}
\label{sec:onderzoeksdeel1}
Dit gaat over de algemene voordelen van zo een pipeline, maar ook specifiek van Amista. Om een antwoord te krijgen op de algemene voordelen wordt er als basis teruggegrepen naar de literatuur. Er is namelijk al enorm veel geschreven over een CI/CD pipeline en wat de voor- en nadelen kunnen zijn.
Bijkomstig is er een interview afgenomen met de expert omtrent DevOps, Patrick Debois.
Het interview en de literatuur bieden samen de perfecte combinatie om de voordelen en nadelen te bespreken.
Op de vraag wat de voor- en nadelen voor Amista zullen zijn kan enkel iemand van Amista op antwoorden natuurlijk. Er zijn vragen gesteld aan Pieter-Jan Deraedt om te kijken hoe Amista denkt te winnen bij het integreren van een CI/CD pipeline.

\section{Onderzoeksdomein 2: Vergelijken van de beschikbare tools}
\label{sec:onderzoeksdeel2}
Wat zijn voor Amista nu de beste tools om een CI/CD pipeline te integreren in hun software development? Welke tools scoren het beste op vlak van snelheid, configureerbaarheid met SAP en de tools die Amista gebruikt, documentatie die te vinden is online en de kostprijs. 

\section{Onderzoeksdomein 3: Voorbeeldapplicatie dat Amista zal gebruiken}
\label{sec:onderzoeksdeel3}
Hoe ziet de omgeving er uit waar een CI/CD pipeline geïntegreerd moet worden er uit? In de literatuurstudie is de uitleg over de tools terug te vinden, maar hier worden de onderliggende relaties tussen de tools besproken.
Er zal een test omgeving opgezet worden die hier besproken zal worden. 

\section{Onderzoeksdomein 4: Handleiding van een CI/CD pipeline voor een SAPUI5 applicatie op SAP Cloud Platform}
\label{sec:onderzoeksdeel4}
Het eigenlijke doel van deze studie is het maken van een mooie 'handleiding' over de best practices om een CI/CD pipeline op te starten met de winnende tools uit Onderzoeksdomein 2. Zo heeft Amista een 'gepersonaliseerde tutorial' hoe een pipeline op te zetten met de tools die het beste matchen met de noden van het bedrijf zelf.



% \lipsum[21-25]

