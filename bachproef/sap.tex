\chapter{\IfLanguageName{dutch}{SAP}{SAP}}
\label{ch:sap}
SAP is een Duitse onderneming opgericht in 1972 dat softwareoplossingen aanbiedt voor grote ondernemingen. Met meer dan 413.000 klanten verspreid over 180 landen mag SAP zich marktleider noemen op gebied van bedrijfssoftware.
SAP heeft zich gespecialiseerd in ERP (Enterprise Resource Planning), software dat alle processen van het bedrijf automatiseert\footnote{https://www.sap.com/products/enterprise-management-erp.html}. Een ERP systeem beheert meerdere functies en bedrijfsprocessen van één bedrijf op basis van een centrale database. Het heeft als doel de gegevens van de organisatie optimaal te gebruiken in de gehele organisatie en een betere beheersing van de bedrijfsprocessen voorzien.
De oplossingen die SAP aanbiedt zijn vooral bedoeld voor de grotere bedrijven. Ze bieden software aan voor elke mogelijke industrie die er vandaag de dag bestaat, maar pakken uit met hun specialiteiten in de cloud business software.
Hieronder worden kort de programma's beschreven die gebruikt worden in de voorbeeldapplicatie. Het geeft een algemeen beeld van de omgeving waar Amista graag een CI/CD pipeline wil voor integreren.

    \paragraph{SAP Cloud Platform}
    SAP Cloud Platform is een Platform as a Service (PaaS), dat aangeboden wordt door SAP. het is een online platform dat - door hardware en software samen te brengen - applicaties overal toegankelijk maakt en samenbrengt tot één platform dat online ter beschikking is\footnote{https://cloudplatform.sap.com/index.html}.
    SAP Cloud Platform wordt zowel voor development als deployment gebruikt, maar reikt ook de hand aan verschillende technologieën: Internet of Things, Big Data, Artificiële Intelligentie enzovoort. Het is een platform dat zowel on-premise - waarbij software enkel lokaal op een computer beschikbaar is - als cloud technologieën samen kan brengen. Je kan er de technologieën ook uitbreiden en zelf ontwikkelen. Het haalt zijn kracht uit de perfecte integratie met andere SAP software die je ook nog eens kan uitbreiden.\\
    Het verschil tussen een Platform as a Service (PaaS), een Infrastructure as a Service (IaaS) en Software as a Service (SaaS) is te vinden in de hoeveelheid er van de leverancier verwacht wordt. SAP Cloud Platform biedt de mogelijkheid tot online developen, testen en deployen van applicaties. Ook de database draait online in de cloud. Het is echter ook mogelijk om de on-premise applicaties te hosten, dit maakt het een perfect voorbeeld van een PaaS. Bij een SaaS omgeving draait alles in de cloud, het is niet mogelijk om ook maar iets on-premise te draaien. Alles wordt geregeld door de leverancier: security, updates, ... .\\
    Bij een IaaS voorziet de leverancier enkel de hardware (in de cloud) aan de klant. Een perfect voorbeeld hiervan is de server die gebruikt wordt in hoofdstuk \ref{ch:methodologie}. Digital Ocean voorziet enkel de Ubuntu server, wat er gebeurd met de server is volledig autonoom te beslissen.
    
    \paragraph{SAPUI5}
    SAPUI5 is een framework dat uitgevonden is door SAP en bevat verschillende libraries die bovenop JavaScript gebouwd zijn. Via het SAP Cloud Platform kunnen er front-end applicaties gemaakt en deployed worden die geschreven zijn in SAPUI5. Het is een framework dat bedoeld is om HTML5 applicaties te bouwen die bijna automatisch responsive zijn zonder veel bijkomende code toe te voegen.
    Het maakt gebruik van dezelfde lay-out zodat het voor de eindklanten één mooi geheel vormt. Het biedt de developers een resem aan UI controls aan, zodat er een consistenter en beter UX design gehanteerd wordt ~\autocite{SAPSEa}. \\
    Mobile en UX design zijn belangrijk geworden doorheen de jaren. SAP heeft daar een antwoord op geboden aan de hand van Fiori. Dit is een term dat gebonden is aan responsive design en OData. Het gebruikt OData (als dataservice) en voorziet eenzelfde lay-out doorheen alle applicaties die tevens ook mobile friendly zijn. Het voorziet ook enkele templates die werken met SAPUI5 zoals een List Report, ... \footnote{https://www.sap.com/products/fiori.html}.
    
    \paragraph{SAP HANA}
    Een In-Memory Data Platform staat als titel op de site van SAP te lezen. Het is een platform dat gebruik maakt van het RAM-geheugen van de computer, wat enorme snelheden met zich meebrengt, maar ook een enorm kostenplaatje. Dit even terzijde wordt SAP HANA aangeprezen als een platform om ingewikkelde, real-time en analytische berekeningen uit te voeren op data.
    Het is een Relationeel Database Management Systeem (RDMBMS) dat geïntegreerd kan worden in SAP Cloud Platform, waarbij het mogelijk is om zowel on-premise als in de cloud te werken, of een combinatie van beiden\footnote{https://www.sap.com/products/hana.html}.
