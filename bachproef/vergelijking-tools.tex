%%=============================================================================
%% Vergelijking van de tools
%%=============================================================================

\chapter{\IfLanguageName{dutch}{Vergelijking van de tools}{Comparing of the tools}}
\label{ch:vergelijking-tools}

\section{Tools voor een CI/CD pipeline}
\label{sec:tools-voor-pipeline}
Er bestaan verschillende repository management services die een source control systeem hebben. GitHub, GitLab en Bitbucket zijn enkele voorbeelden.
Build schedulers zorgen ervoor dat de procedures worden samengesteld en dat de builds worden getriggerd. Voorbeelden hiervan zijn: Jenkins, Travis CI, GitLab-CI en Bamboo. Sonatype Nexus en Archiva zijn dan weer voorbeelden van tools die gebruikt worden als artifact repository manager, deze houden bij wijze van spreken de code bij die klaar is om te deployen. 

\paragraph{Source Control System}
Dit systeem houdt alle veranderingen bij aan de code en zal de veranderingen ook beheren zodat ze niet overlappen. Het laat toe dat meerdere developers tegelijk aan - soms dezelfde - code werken. Een source control systeem houdt dan de veranderingen bij wat elke developer gedaan heeft. Best practice is er maar 1 versie van de software die stabiel is, meestal master genoemd. Een gangbare praktijk binnen dit systeem is het maken van branches. Hier wordt een kopie gemaakt van de stabiele master, waardoor men wat kan knoeien in de branch zonder de master te beschadigen. Als men overtuigd is van de kwaliteiten dat men op een branch gemaakt heeft kan men de branch 'mergen' in de master. Deze techniek wordt ook wel 'revision control' of 'version control' genoemd ~\autocite{Skelton2014} en ~\autocite{Riti2018}.
Voorbeelden van zo een source control system zijn Git, CVS (Concurrent Version System), Subversion en Mercurial.

\paragraph{Repository Management Services}
Wordt ook wel Code Repository Server genoemd. Hier wordt de software van het source control system opgeslagen. Dit kan op een interne server opgeslagen worden, of op een externe die door bedrijven aangeboden wordt. Dit maakt het makkelijker voor developers om samen te werken en dezelfde bron te gebruiken en is nodig om een goede Continuous Integration aan te bieden.

\paragraph{Build Scheduler}
Een Build Scheduler kan ook wel een Continuous Integration Server genoemd worden.
Dit zorgt ervoor dat - telkens wanneer er code gecommit wordt - de pipeline wordt uitgevoerd. De taken van de build scheduler zijn: 
\begin{itemize}
    \item Code ophalen van de Repository Server en deze samenvoegen met de oude code
    \item De testen uitvoeren
    \item Het builden van de software
    \item Feedback geven aan de developer over voorgaande stappen
\end{itemize}
Deze taken kunnen ook door een script volbracht worden, maar het is belangrijk dat deze taken automatisch gebeuren telkens er code gecommit wordt naar de repository manager ~\autocite{Riti2018}.

\paragraph{Artifact Repository Manager}
Als laatste zijn de Artifact Repository Managers aan de beurt. Deze repository manager houdt alles wat nodig is om de applicatie te deployen bij zoals
\begin{itemize}
    \item packaged application code
    \item application assets
    \item infrastructure code
    \item virtual machine images
    \item configuration data
\end{itemize}
Deze tool houdt alle geschiedenis bij van de bovenstaande files. Zoals eerder vermeld kan men vanaf hier de software (automatisch) deployen. Dit is de allerlaatste fase in de CI/CD pipeline ~\autocite{Skelton2014}.
