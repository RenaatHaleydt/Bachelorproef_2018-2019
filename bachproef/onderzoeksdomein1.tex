%%=============================================================================
%% Onderzoeksdomein1
%%=============================================================================

\chapter{\IfLanguageName{dutch}{Onderzoeksdomein 1: voor- en nadelen CI/CD pipeline}{Research field 1: benefits and drawbacks of a CI/CD pipeline}}
\label{ch:onderzoeksdomein1}
De waarom-vraag is minstens even belangrijk als de hoe-vraag. Men mag niet vergeten waarom men een CI/CD pipeline wil opzetten om tot een goed eindresultaat te komen. Concreet: wat zijn de voor- en nadelen van een CI/CD pipeline te integreren in het algemeen en specifiek voor Amista?
Een antwoord op deze vraag komt uit de literatuur, maar ook van Amista.

\section{Voordelen}
\label{sec:voordelen}
\begin{itemize}
    \item Als men kleine stukjes code commit naar de centrale repository en de testen slagen niet, weet men dat het aan dit klein deeltje van de code ligt en kan men de fout veel sneller opsporen.
    \item In de meeste omgevingen met een CI/CD pipeline is het niet nodig om een QA team te hebben om de testen uit te voeren, wat de kosten doet dalen natuurlijk.
\end{itemize}

\section{Nadelen}
\label{sec:nadelen}
