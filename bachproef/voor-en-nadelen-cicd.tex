%%=============================================================================
%% Onderzoeksdomein1
%%=============================================================================

\chapter{\IfLanguageName{dutch}{Voor- en nadelen CI/CD pipeline}{Research field 1: benefits and drawbacks of a CI/CD pipeline}}
\label{ch:voor-en-nadelen-cicd}
Dit gaat over de algemene voordelen van zo een pipeline, maar ook specifiek van Amista. Om een antwoord te krijgen op de algemene voordelen wordt er als basis teruggegrepen naar de literatuur. Er is namelijk al enorm veel geschreven over een CI/CD pipeline en wat de voor- en nadelen kunnen zijn.
Bijkomstig is er een interview afgenomen met de expert omtrent DevOps, Patrick Debois.
Het interview en de literatuur bieden samen de perfecte combinatie om de voordelen en nadelen te bespreken.
Op de vraag wat de voor- en nadelen voor Amista zullen zijn kan enkel iemand van Amista op antwoorden.


\section{Voordelen}
\label{sec:voordelen}
\begin{itemize}
    \item Als men kleine stukjes code commit naar de centrale repository en de testen slagen niet, weet men dat het aan dit klein deeltje van de code ligt en kan men de fout veel sneller opsporen. %TODO: Zie ~\autocite{Fowler2006}, daar staat staat in hoofdstuk Everyone Commits To the Mainline Every Day. Hiernaar kan ik verwijzen voor dit voordeel.
    \item In de meeste omgevingen met een CI/CD pipeline is het niet nodig om een QA team te hebben om de testen uit te voeren, wat de kosten doet dalen.
\end{itemize}

\section{Nadelen}
\label{sec:nadelen}
