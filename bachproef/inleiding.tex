%%=============================================================================
%% Inleiding
%%=============================================================================

\chapter{\IfLanguageName{dutch}{Inleiding}{Introduction}}
\label{ch:inleiding}

Een Continuous Integration en Continuous Delivery (CI/CD) pipeline opzetten voor de ontwikkeling van software is zeer hip vandaag. Zo een pipeline zorgt voor de automatisatie van testen, builds en deployment en heeft als grote voordeel dat er sneller wijzigingen doorgevoerd kunnen worden. Bij Amista weten ze dat er veel potentieel zit in een CI/CD pipeline. Amista is een consultancy bedrijf dat gespecialiseerd is in SAP. Ze leveren op maat gemaakte oplossingen aan grote en middelgrote bedrijven die hun processen geautomatiseerd willen maken. 

Er is niet veel documentatie te vinden om een CI/CD pipeline op te zetten binnen SAP.
Deze thesis biedt een leidraad om zo een pipeline op te zetten binnen een SAPUI5 webapplicatie dat verbonden is met SAP HANA en gehost wordt op SAP Cloud Platform. 
Dit zijn tevens ook de zaken waar ze bij Amista op zoek naar zijn en biedt dus een op maat gemaakte handleiding voor Amista aan.

Om een Continuous Integration en Continuous Delivery pipeline op te stellen zijn er vandaag veel tools beschikbaar.
Het doel van deze thesis is ook om de beste tools uit te kiezen op vlak van snelheid, hoeveelheid documentatie, configureerbaarheid met de huidige tools die Amista gebruikt en kostprijs.

Aan de hand van een voorbeeld applicatie zou deze thesis een antwoord bieden op de vragen waarom en hoe een Continuous Integration en Continuous Delivery pipeline op te zetten.



%De inleiding moet de lezer net genoeg informatie verschaffen om het onderwerp te begrijpen en in te zien waarom de onderzoeksvraag de moeite waard is om te onderzoeken. In de inleiding ga je literatuurverwijzingen beperken, zodat de tekst vlot leesbaar blijft. Je kan de inleiding verder onderverdelen in secties als dit de tekst verduidelijkt. Zaken die aan bod kunnen komen in de inleiding~\autocite{Pollefliet2011}:
%
%\begin{itemize}
%  \item context, achtergrond
%  \item afbakenen van het onderwerp
%  \item verantwoording van het onderwerp, methodologie
%  \item probleemstelling
%  \item onderzoeksdoelstelling
%  \item onderzoeksvraag
%  \item \ldots
%\end{itemize}

\section{\IfLanguageName{dutch}{Probleemstelling}{Problem Statement}}
\label{sec:probleemstelling}

Zoals eerder aangegeven is vandaag bijna geen documentatie beschikbaar hoe een CI/CD pipeline te implementeren in een SAPUI5 webapplicatie met SAP HANA dat gebruik maakt van Node.js gehost op SAP Cloud Platform. 
Daarom is Amista op zoek naar een voorbeeld applicatie om de technieken uit te rollen voor hun dagelijkse werking.
%Uit je probleemstelling moet duidelijk zijn dat je onderzoek een meerwaarde heeft voor een concrete doelgroep. De doelgroep moet goed gedefinieerd en afgelijnd zijn. Doelgroepen als ``bedrijven,'' ``KMO's,'' systeembeheerders, enz.~zijn nog te vaag. Als je een lijstje kan maken van de personen/organisaties die een meerwaarde zullen vinden in deze bachelorproef (dit is eigenlijk je steekproefkader), dan is dat een indicatie dat de doelgroep goed gedefinieerd is. Dit kan een enkel bedrijf zijn of zelfs één persoon (je co-promotor/opdrachtgever).

\section{\IfLanguageName{dutch}{Onderzoeksvraag}{Research question}}
\label{sec:onderzoeksvraag}

\begin{itemize}
    \item Wat zijn de voor- en nadelen van een CI/CD pipeline te integreren in het algemeen en specifiek voor Amista?
    \item Is het mogelijk om op een eenvoudige manier een Continuous Integration en Continuous Delivery pipeline te implementeren voor de ontwikkelingen van een SAPUI5 applicatie op SAP Cloud Platform?
    \item Hoe kunnen we deze implementatie tot een succes brengen?
    \item Welke tools moeten we gebruiken om een CI/CD pipeline op SAP Cloud Platform te implementeren als we vergelijken op snelheid, configureerbaarheid, documentatie en kostprijs.
\end{itemize}
%Wees zo concreet mogelijk bij het formuleren van je onderzoeksvraag. Een onderzoeksvraag is trouwens iets waar nog niemand op dit moment een antwoord heeft (voor zover je kan nagaan). Het opzoeken van bestaande informatie (bv. ``welke tools bestaan er voor deze toepassing?'') is dus geen onderzoeksvraag. Je kan de onderzoeksvraag verder specifiëren in deelvragen. Bv.~als je onderzoek gaat over performantiemetingen, dan 

\section{\IfLanguageName{dutch}{Onderzoeksdoelstelling}{Research objective}}
\label{sec:onderzoeksdoelstelling}

Deze thesis zou als hoofddoel een proof-of-concept zijn hoe een CI/CD pipeline te gebruiken in de dagelijkse werking van Amista. 
Maar deze thesis gaat ook op zoek naar de beste tools om de pipeline op te bouwen rekening houdend met snelheid, configureerbaarheid met de tools die Amista gebruikt, hoeveelheid documentatie en kostprijs.
%Wat is het beoogde resultaat van je bachelorproef? Wat zijn de criteria voor succes? Beschrijf die zo concreet mogelijk. Gaat het bv. om een proof-of-concept, een prototype, een verslag met aanbevelingen, een vergelijkende studie, enz.

\section{\IfLanguageName{dutch}{Opzet van deze bachelorproef}{Structure of this bachelor thesis}}
\label{sec:opzet-bachelorproef}

% Het is gebruikelijk aan het einde van de inleiding een overzicht te
% geven van de opbouw van de rest van de tekst. Deze sectie bevat al een aanzet
% die je kan aanvullen/aanpassen in functie van je eigen tekst.

De rest van deze bachelorproef is als volgt opgebouwd:

In Hoofdstuk~\ref{ch:stand-van-zaken} wordt een overzicht gegeven van de stand van zaken binnen het onderzoeksdomein, op basis van een literatuurstudie.

In Hoofdstuk~\ref{ch:methodologie} wordt de methodologie toegelicht en worden de gebruikte onderzoekstechnieken besproken om een antwoord te kunnen formuleren op de onderzoeksvragen.

% TODO: Vul hier aan voor je eigen hoofstukken, één of twee zinnen per hoofdstuk


In Hoofdstuk~\ref{ch:conclusie}, tenslotte, wordt de conclusie gegeven en een antwoord geformuleerd op de onderzoeksvragen. Daarbij wordt ook een aanzet gegeven voor toekomstig onderzoek binnen dit domein.