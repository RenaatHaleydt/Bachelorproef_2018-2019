%%=============================================================================
%% Inleiding
%%=============================================================================

\chapter{\IfLanguageName{dutch}{Inleiding}{Introduction}}
\label{ch:inleiding}

It-bedrijven willen hun klanten sneller te hulp kunnen schieten, betere software leveren, maar worstelen nog vaak met deadlines die verstrijken en veel problemen tijdens de oplevering van software. Gelukkig is er doorheen de jaren al veel progressie gemaakt door principes zoals DevOps bijvoorbeeld toe te passen, een mindset om de samenwerking tussen alle afdelingen binnen een it-bedrijf vlotter te maken. 
Een Continuous Integration en Continuous Delivery (CI/CD) pipeline opzetten is een onderdeel van DevOps en zou de werking van een bedrijf ten goede komen. Een CI/CD pipeline zorgt voor de automatisatie van testen, builds en deployment en heeft als grote voordeel dat er sneller wijzigingen doorgevoerd kunnen worden.

Amista is een groeiend bedrijf binnen IT consultancy. Het is een dochterbedrijf van Boutique dat zich toespitst op SAP. Sinds hun vijfjarig bestaan werken er al 60 personen voor Amista. Ze hebben Infrabel, Alcopa, Danone, AG Real Estate en nog anderen tot hun cliënteel, zoals op hun site te lezen valt ~\autocite{Amista2018}.
Ze zijn in twee landen actief, Frankrijk en België, maar werken ook samen met mensen uit India.
Amista heeft zich verdiept in de Sales, Marketing en Service Management takken van bedrijven. Ze helpen hun klanten ook op gebied van Innovation, Integration, HCM Services (implementatie van succesfactoren binnen de HR afdeling) en Digital Learning.
De missie van Amista is om samen met hun klanten innovatieve en kwalitatieve oplossingen aan te bieden met behulp van het volledige gamma dat SAP te bieden heeft.
Omdat Amista graag de wensen van hun klanten zo goed mogelijk probeert te vervullen, proberen ze zelf zo innovatief mogelijk te zijn. Ze spelen met het idee om een CI/CD pipeline op te zetten voor de software development.

Om een Continuous Integration en Continuous Delivery pipeline op te stellen zijn er vandaag veel tools beschikbaar.
Deze thesis zal verschillende build schedulers vergelijken op basis van snelheid, hoeveelheid beschikbare documentatie en configureerbaarheid met de huidige set-up van Amista.
Aan de hand van de winnende build scheduler wordt er een handleiding beschreven om een pipeline op te zetten binnen een SAPUI5 webapplicatie dat verbonden is met SAP HANA en gehost wordt op SAP Cloud Platform.

%De inleiding moet de lezer net genoeg informatie verschaffen om het onderwerp te begrijpen en in te zien waarom de onderzoeksvraag de moeite waard is om te onderzoeken. In de inleiding ga je literatuurverwijzingen beperken, zodat de tekst vlot leesbaar blijft. Je kan de inleiding verder onderverdelen in secties als dit de tekst verduidelijkt. Zaken die aan bod kunnen komen in de inleiding~\autocite{Pollefliet2011}:
%
%\begin{itemize}
%  \item context, achtergrond
%  \item afbakenen van het onderwerp
%  \item verantwoording van het onderwerp, methodologie
%  \item probleemstelling
%  \item onderzoeksdoelstelling
%  \item onderzoeksvraag
%  \item \ldots
%\end{itemize}

\section{\IfLanguageName{dutch}{Probleemstelling}{Problem Statement}}
\label{sec:probleemstelling}

Amista heeft enkele klanten waar ze SAPUI5 webapplicaties voor moeten maken. Deze combineren ze met een SAP HANA database dat allemaal draait op SAP Cloud Platform. Omdat er nog niet veel informatie te vinden is over een CI/CD pipeline binnen bovenstaande toepassingen, wil Amista heel graag een gepersonaliseerde handleiding om zo een pipeline op te stellen.
Zoals eerder aangegeven is vandaag bijna geen documentatie beschikbaar hoe een CI/CD pipeline te implementeren in een SAPUI5 webapplicatie met SAP HANA dat gebruik maakt van Node.js gehost op SAP Cloud Platform. 
Daarom is Amista op zoek naar een voorbeeld applicatie om de technieken uit te rollen voor hun dagelijkse werking.
%Uit je probleemstelling moet duidelijk zijn dat je onderzoek een meerwaarde heeft voor een concrete doelgroep. De doelgroep moet goed gedefinieerd en afgelijnd zijn. Doelgroepen als ``bedrijven,'' ``KMO's,'' systeembeheerders, enz.~zijn nog te vaag. Als je een lijstje kan maken van de personen/organisaties die een meerwaarde zullen vinden in deze bachelorproef (dit is eigenlijk je steekproefkader), dan is dat een indicatie dat de doelgroep goed gedefinieerd is. Dit kan een enkel bedrijf zijn of zelfs één persoon (je co-promotor/opdrachtgever).

\section{\IfLanguageName{dutch}{Onderzoeksvraag}{Research question}}
\label{sec:onderzoeksvraag}

\begin{itemize}
    \item Wat zijn de voor- en nadelen van een CI/CD pipeline te integreren in het algemeen en specifiek voor Amista?
    \item Is het mogelijk om op een eenvoudige manier een Continuous Integration en Continuous Delivery pipeline te implementeren voor de ontwikkelingen van een SAPUI5 applicatie op SAP Cloud Platform?
    \item Hoe kunnen we deze implementatie tot een succes brengen?
    \item Welke tools moeten we gebruiken om een CI/CD pipeline op SAP Cloud Platform te implementeren als we vergelijken op snelheid, configureerbaarheid, documentatie en kostprijs.
\end{itemize}
%Wees zo concreet mogelijk bij het formuleren van je onderzoeksvraag. Een onderzoeksvraag is trouwens iets waar nog niemand op dit moment een antwoord heeft (voor zover je kan nagaan). Het opzoeken van bestaande informatie (bv. ``welke tools bestaan er voor deze toepassing?'') is dus geen onderzoeksvraag. Je kan de onderzoeksvraag verder specifiëren in deelvragen. Bv.~als je onderzoek gaat over performantiemetingen, dan 

\section{\IfLanguageName{dutch}{Onderzoeksdoelstelling}{Research objective}}
\label{sec:onderzoeksdoelstelling}

Deze thesis zou als hoofddoel een proof-of-concept zijn hoe een CI/CD pipeline te gebruiken in de dagelijkse werking van Amista. 
Maar deze thesis gaat ook op zoek naar de beste tools om de pipeline op te bouwen rekening houdend met snelheid, configureerbaarheid met de tools die Amista gebruikt, hoeveelheid documentatie en kostprijs.
%Wat is het beoogde resultaat van je bachelorproef? Wat zijn de criteria voor succes? Beschrijf die zo concreet mogelijk. Gaat het bv. om een proof-of-concept, een prototype, een verslag met aanbevelingen, een vergelijkende studie, enz.

\section{\IfLanguageName{dutch}{Opzet van deze bachelorproef}{Structure of this bachelor thesis}}
\label{sec:opzet-bachelorproef}

% Het is gebruikelijk aan het einde van de inleiding een overzicht te
% geven van de opbouw van de rest van de tekst. Deze sectie bevat al een aanzet
% die je kan aanvullen/aanpassen in functie van je eigen tekst.

De rest van deze bachelorproef is als volgt opgebouwd:

In Hoofdstuk~\ref{ch:stand-van-zaken} wordt een overzicht gegeven van de stand van zaken binnen het onderzoeksdomein, op basis van een literatuurstudie.

In Hoofdstuk~\ref{ch:methodologie} wordt de methodologie toegelicht en worden de gebruikte onderzoekstechnieken besproken om een antwoord te kunnen formuleren op de onderzoeksvragen.

De set-up van de omgeving wordt in detail besproken in Hoofdstuk~\ref{ch:voorbeeldapplicatie}.

Een vergelijking tussen de verschillende build schedulers kan men vinden in Hoofdstuk~\ref{ch:vergelijking-tools}

In Hoofdstuk~\ref{ch:handleiding} wordt er een gepersonaliseerde handleiding gegeven om met de beste build scheduler een CI/CD pipeline te implementeren.

De waarom-vraag is minstens even belangrijk als de hoe-vraag. Concreet: wat zijn de voor- en nadelen van een CI/CD pipeline te integreren in het algemeen en specifiek voor Amista? Een antwoord op deze vragen kan je in Hoofdstuk~\ref{ch:voor-en-nadelen-cicd}.

In Hoofdstuk~\ref{ch:conclusie}, tenslotte, wordt de conclusie gegeven en een antwoord geformuleerd op de onderzoeksvragen. Daarbij wordt ook een aanzet gegeven voor toekomstig onderzoek binnen dit domein.