%%=============================================================================
%% Conclusie
%%=============================================================================

\chapter{Conclusie}
\label{ch:conclusie}

% TODO: Trek een duidelijke conclusie, in de vorm van een antwoord op de
% onderzoeksvra(a)g(en). Wat was jouw bijdrage aan het onderzoeksdomein en
% hoe biedt dit meerwaarde aan het vakgebied/doelgroep? 
% Reflecteer kritisch over het resultaat. In Engelse teksten wordt deze sectie
% ``Discussion'' genoemd. Had je deze uitkomst verwacht? Zijn er zaken die nog
% niet duidelijk zijn?
% Heeft het onderzoek geleid tot nieuwe vragen die uitnodigen tot verder 
%onderzoek?

%\lipsum[76-80]

Uit onderzoek blijkt dat de inspanningen om een Continuous Integration en Continuous Delivery pipeline op te zetten niet opwegen tegen de vele voordelen ervan. Een goed uitgewerkte pipeline zal de frustraties bij klanten en developers weghalen en komt de kwaliteit van de code ten goede.

Het moet echter mogelijk zijn om een pipeline binnen de set-up van een bedrijf te hanteren. De set-up die ze bij Amista hanteren, is niet alomtegenwoordig wat de moeilijkheidsgraad doet stijgen. \\
Voor bedrijven die willen inzetten op de nieuwe technologieën binnen SAP, een SAPUI5 webapplicatie in combinatie met een SAP HANA 2.0 database dat gebruik maakt van Node.js en wordt gehost op SAP Cloud Platform, en deze willen integreren in een CI/CD pipeline, biedt deze bachelorproef een wegwijzer door het avontuur.\\

%TODO: kiezen!
ALS HET NIET SLAAGT:\\
Op de vraag of het mogelijk is om een CI/CD pipeline op te zetten binnen deze specifieke set-up? Deze studie maakt duidelijk dat het tot op een bepaald punt mogelijk is. Een Continuous Integration implementeren is geen probleem. Een Continuous Delivery pipeline voorzien is een heel ander paar mouwen.

Uit de vergelijkende studie tussen de build schedulers komt Jenkins er als grote winnaar uit. Jenkins scoorde op de meeste vlakken het beste en bleek de ultieme tool om deze studie rond te bouwen. Zo bleek Jenkins bruikbaar met SAPUI5 en SAP Cloud Platform en kon het integreren met Git en BitBucket. Nog een grote doorslaggevende factor was de integratie met Karma.

Niet alleen het opzetten van de projecten om de pipeline op te bouwen worden besproken, maar ook de QUnit en OPA5 tests worden automatisch gerund dankzij Karma. De handleiding wordt aangevuld met het opzetten van de Jenkins server en de Continuous Integration omgeving.

Zoals verwacht kwamen er heel wat problemen naar boven tijdens het opzetten van de CI/CD pipeline. Tot op een bepaald punt zijn deze allemaal gecounterd, maar bepaalde zaken zijn niet opgelost, zoals het opzetten van de Continuous Delivery pipeline.
Het blijft bij een uitgebreide handleiding voor Continuous Integration en een algemene gids voor Continuous Delivery. Dit heeft als reden dat de specifieke set-up om de SAPUI5 applicatie te deployen op SAP Cloud Platform heel moeilijk is om op te zetten.

De vraag die na deze studie overblijft is: hoe lukt het wel om een Continuous Delivery pipeline op te zetten? Is er een betere tool beschikbaar om de problemen te counteren of moet de SAPUI5 applicatie worden gehost op Cloud Foundry?

ALS HET SLAAGT:\\
Op de vraag of het mogelijk is om een CI/CD pipeline op te zetten binnen deze specifieke set-up? Deze studie maakt duidelijk dat het effectief mogelijk is.

Uit de vergelijkende studie tussen de build schedulers komt Jenkins er als grote winnaar uit. Jenkins scoorde op de meeste vlakken het beste en bleek de ultieme tool om deze studie rond te bouwen. Zo bleek Jenkins bruikbaar met SAPUI5 en SAP Cloud Platform en kon het integreren met Git en BitBucket. Nog een grote doorslaggevende factor was de integratie met Karma.

Zoals verwacht kwamen er heel wat problemen naar boven tijdens het opzetten van de CI/CD pipeline. Deze bachelorproef biedt echter een goede gids om de nieuwe technologieën te combineren met de pipeline.\\
Niet alleen het opzetten van de projecten om de pipeline op te bouwen worden besproken, maar ook de QUnit en OPA5 tests worden automatisch gerund dankzij Karma. De handleiding wordt aangevuld met het opzetten van de Jenkins server en de Continuous Integration omgeving. Bovendien biedt de handleiding ook een antwoord op de vraag hoe een Continuous Delivery pipeline op te zetten binnen deze set-up.

De algemene conclusie is dat het mogelijk is om een Continuous Integration en Continuous Delivery pipeline op te bouwen met technologieën zoals SAPUI5 en SAP Cloud Platform. Het blijft echter een huzarenstukje om dit voor elkaar te spelen, maar deze bachelorproef biedt een mooie leidraad bij dit avontuur.