%%=============================================================================
%% Technische vergelijking tussen build schedulers
%%=============================================================================

\chapter{\IfLanguageName{dutch}{Vergelijking tussen build schedulers}{Comparing build schedulers}}
\label{ch:vergelijking-build-schedulers}
Hier worden de verschillende criteria die Amista aangaf als belangrijk en minder belangrijk opgedeeld in twee delen: de functionele requirements en de niet-functionele requirements. Binnen beide categorieën wordt er nog een opsplitsing gemaakt tussen: must-haves, should-haves en nice-to-haves.
Met deze informatie kunnen we al een eerste vergelijking maken tussen de build-schedulers die vandaag op de markt te vinden zijn. Als resultaat krijgen we een long list waar de tools naast elkaar worden gezet en vergeleken kunnen worden aan de hand van criteria. 
Uit deze eerste vergelijking nemen we de beste tools eruit om te testen in een realistische omgeving. Dit zal in hoofdstuk\ref{ch:proof-of-concept} gebeuren.

\section{Functionele requirements}
\label{sec:functionele-requirements}

\section{Niet-functionele requirements}
\label{sec:niet-functionele-requirements}

\section{Must-Haves}
\label{sec:must-haves}

\section{Should-Haves}
\label{sec:should-haves}

\section{Nice-to-Haves}
\label{sec:nice-to-haves}

\section{Long list}
\label{sec:Long-list}




