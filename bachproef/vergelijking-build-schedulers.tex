%%=============================================================================
%% Technische vergelijking tussen build schedulers
%%=============================================================================


\chapter{\IfLanguageName{dutch}{Vergelijking tussen build schedulers}{Comparing build schedulers}}
\label{ch:vergelijking-build-schedulers}
Hier worden de verschillende criteria die Amista aangaf als belangrijk en minder belangrijk opgedeeld in twee delen: de functionele requirements en de niet-functionele requirements. Binnen beide categorieën wordt er nog een opsplitsing gemaakt tussen: must-haves, should-haves en nice-to-haves.
Met deze informatie kunnen we al een eerste vergelijking maken tussen de build-schedulers die vandaag op de markt te vinden zijn. Als resultaat krijgen we een long list waar de tools naast elkaar worden gezet en vergeleken kunnen worden aan de hand van criteria. 
Uit deze eerste vergelijking nemen we de beste tools eruit om te testen in een realistische omgeving. Dit zal in hoofdstuk\ref{ch:proof-of-concept} gebeuren.

\section{Requirements}
\label{sec:requirements}
Hier zullen de twee soorten requirements met elkaar vergeleken worden.

\paragraph{Functionele requirements}
Een functionele requirement is een functie wat het systeem moet doen.
De punten die in deze sectie worden aangehaald komen uit hoofdstuk\ref{ch:voor-en-nadelen-cicd}. Om een duidelijk beeld te krijgen wat de build scheduler moet doen worden de functionele requirements hier nog eens kort besproken.
\begin{itemize}
    \item De build scheduler moet de aanpassingen aan de code sneller naar de klant brengen
    \item De downtime van een applicatie moet dalen
    \item Er moet vermeden worden dat een applicatie stopt met werken omdat er een fout in de code zit die gedeployed wordt
\end{itemize}

\paragraph{Niet-functionele requirements}
Een niet-functionele requirement legt uit hoe het systeem een bepaalde functie moet uitvoeren. Voor Amista is het belangrijk dat de build scheduler voldoet aan hoge security eisen. Vaak wordt er met hele grote projecten gewerkt die gevoelige data en broncode beschikken. Om optimale veiligheid te garanderen wordt gewerkt met een build scheduler die op een on-premise systeem zal draaien. Een andere mogelijkheid, waar vandaag de dag veel in wordt geïnvesteerd is een cloud build scheduler. Dit zorgt ervoor dat de code buiten de onderneming gaat, wat toch een zeker risico met zich meebrengt.


\section{Criteria}
\label{sec:criteria}

\paragraph{Must-haves}
De must-haves van de build scheduler zullen vooral te maken hebben met de omgeving waarin ze moeten werken. Zoals in de literatuurstudie al beschreven staat, moet een build scheduler gebruikt worden met een SAPUI5 applicatie, samen met een SAP HANA database gehost op SAP Cloud Platform.
In deze thesis wordt dezelfde source control manager gebruikt als bij Amista. Het is niet van toepassing om zelf een source control system en repository manager te kiezen, er moet gebruik gemaakt worden van Git en Bitbucket. Het is vanzelfsprekend dat de build scheduler moet kunnen integreren met Git en Bitbucket.
Om een betere vergelijking te kunnen maken heeft Amista een Ubuntu server voorzien voor de uitwerking van de proof-of-concept, zie hoofdstuk\ref{ch:proof-of-concept}. Deze dient ook als simulatie voor de realistische on-premise set-up. Het is dus noodzakelijk dat de build scheduler op een Unix-based systeem kan draaien.
Omdat ze bij Amista met Bitbucket werken is het vanzelfsprekend dat de build scheduler ook kan integreren met deze Code Repository.
De build scheduler moet ook deel kunnen uitmaken van een Continuous Delivery pipeline.

Als we bovenstaande punten even kort samenvatten moet de build scheduler aan volgende vereisten voldoen:
\begin{itemize}
    \item Hij moet bruikbaar zijn met een SAPUI5 applicatie, een SAP HANA database gehost op SAP Cloud Platform
    \item Hij moet kunnen integreren met Git \& Bitbucket
    \item De build scheduler moet op een Unix-based server kunnen draaien
\end{itemize}


\paragraph{Should-Haves}
De build scheduler zou moeten beschikken over gratis gebruik van de software om te experimenteren. Het is de bedoeling dat we in het volgende hoofdstuk een proof-of-concept kunnen uitwerken om bepaalde build schedulers te installeren. Daarom is het nodig om een gratis versie te hebben die de opstelling mogelijk maakt.


\paragraph{Nice-to-Haves}
De build scheduler moet niet per se op een cloud draaien, het belangrijkste is dat de data lokaal gehouden kan worden door de build scheduler op een on-premise installatie op te zetten. Maar Amista wil graag deze optie wel openhouden voor de toekomst. Daarom is het mooi meegenomen als de gekozen build scheduler aan deze vereiste voldoet, maar het is geen breekpunt.
%TODO: Misschien moet er later ook gebruik gemaakt worden van Docker om de builds te bouwen? Dit kan dan ook een vereiste zijn?

\section{Long list}
\label{sec:Long-list}
Nu de verschillende criteria gekend zijn kan de effectieve vergelijking tussen de build schedulers gebeuren.
Eerst zal er een korte uitleg gegeven worden over welke programma's we gaan vergelijken en nadien vindt de vergelijking plaats in een overzichtelijke tabel.

\paragraph{Jenkins}
Jenkins is de meest gekende build scheduler binnen de IT-wereld. Het is een op zichzelf staande, open source automation server dat ontstaan is in 2011 na een afscheiding van het Hudson project. Jenkins is vooral bekend om de vele plug-ins die het mogelijk maken om met bijna alle talen en platformen te integreren. Jenkins is geschreven in Java draait op een Java platform.

\paragraph{Circle CI}
Circle CI is opgericht in 2011 in San Francisco en wordt door vele grote bedrijven gebruikt in hun Continuous Integration pipeline. Het grootste deel van Circle CI is geschreven in Clojure en de Frontend in ClojureScript. 

\paragraph{Bamboo}
Bamboo, een Java-based Continuous Integration en Continuous Delivery tool, werd opgericht in 2007 door het bedrijf Atlassian dat ook Bitbucket onder zijn armen heeft.

\paragraph{Travis CI}
Travis CI is een integration tool dat geschreven is in Ruby en opgericht in 2011 in Duitsland. Het staat bekend om zijn samenwerking met GitHub.

%TODO: https://www.tablesgenerator.com
\begin{table}[]
    \centering
    \begin{tabular}{|l|l|l|l|l|}
        \hline
        \textbf{Build scheduler} & \textbf{\begin{tabular}[c]{@{}l@{}}SAPUI5, HANA \&\\ Cloud Platform\end{tabular}} & \textbf{\begin{tabular}[c]{@{}l@{}}Git \& \\ Bitbucket\end{tabular}} & \textbf{\begin{tabular}[c]{@{}l@{}}Unix-based\\ server\end{tabular}} & \textbf{Cloud toepassing} \\ \hline
        Bamboo &  & Yes &  &  \\ \hline
        Circle CI &  &  &  &  \\ \hline
        Jenkins &  & Yes & Yes &  \\ \hline
        Travis &  &  &  &  \\ \hline
    \end{tabular}
\end{table}




