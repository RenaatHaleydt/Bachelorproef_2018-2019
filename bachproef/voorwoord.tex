%%=============================================================================
%% Voorwoord
%%=============================================================================

\chapter*{\IfLanguageName{dutch}{Woord vooraf}{Preface}}
\label{ch:voorwoord}

%% TODO:
%% Het voorwoord is het enige deel van de bachelorproef waar je vanuit je
%% eigen standpunt (``ik-vorm'') mag schrijven. Je kan hier bv. motiveren
%% waarom jij het onderwerp wil bespreken.
%% Vergeet ook niet te bedanken wie je geholpen/gesteund/... heeft

Deze bachelorproef betekent de eindsprint van mijn opleiding Toegepaste Informatica. Drie jaar lang heb ik het beste van mezelf gegeven. Als dank neem ik een rugzak vol kennis mee naar de echte uitdaging: het werkveld.
Ik wil graag iedere leerkracht bedanken voor het vullen van deze rugzak en voor de hulp gedurende mijn opleiding.

Toen ik bij Amista ging aankloppen om een onderwerp te kiezen voor mijn bachelorproef stelden zij voor om een Continuous Integration en Continuous Delivery pipeline te maken voor SAPUI5 applicaties op SAP Cloud Platform. Ik had geen idee hoe deze termen met elkaar konden worden gelinkt, maar gaandeweg heb ik me verdiept en ben tot de mening gekomen dat dit de toekomst is waar we naartoe moeten gaan in development.

Graag zou ik Amista willen bedanken voor het vertrouwen en de kansen die ze mij aanboden.
Pieter-Jan Deraedt verdient ook een bedanking in het bijzonder. Dit voor de raad, het vertrouwen, de feedback, het geduld en de mogelijkheden die hij me gaf.

Mijn promotor Harm De Weirdt verdient een speciale bedanking. Dit voor het vele nalezen en geven van feedback, de uitstekende begeleiding om deze bachelorproef tot een goed einde te brengen en de steun.

Mijn familie zou ik graag bedanken voor alle steun en vertrouwen gedurende drie jaar.
Tot slot wil ik nog iemand speciaal bedanken: mijn vriendin, Sarah De Visscher. Niet alleen voor het vertrouwen en de steun gedurende mijn opleiding, maar ook voor het nalezen en verbeteren van deze bachelorproef.

